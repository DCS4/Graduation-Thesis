% !TeX root = ../Template.tex
% 总结


\summary

\section*{论文总结}


本文首先阐述了利用RGB-D相机进行三维重建与纹理优化的背景与意义,以及纹理优化领域的研究现状。然后,对纹理优化相关技术做了梳理与总结。针对目前纹理优化中纹理存在模糊伪影以及裂缝现象,本文提出基于深度学习的纹理优化方案,即利用可微分渲染技术矫正相机位姿,借助像素重生成管线重新生成新的纹理图像。为了保证在几何模型重建误差较大的情况下仍能获得高保真的纹理,本文进一步提出了基于可微分渲染的联合优化算法,它可以对初始估计的相机位姿、重建模型、纹理图集进行迭代优化,得到具有高频几何细节和清晰纹理贴图的模型。综上所述,本文贡献点有两个。\par
提出基于可微分渲染的相机位姿矫正与纹理合成算法,获得清晰纹理。首先利用渲染将初始纹理与三维模型渲染至目标视角,通过与真实图片做$L_1$损失,获得期望梯度以更新相机位姿。然后重新初始化纹理图并将纹理图设置为优化变量,利用对抗生成网络重合成清晰的纹理。经过纹理重合成,得到全局一致高保真纹理模型。\par
提出联合优化算法同时对RGB-D重建的相机位姿、几何细节和纹理细节进行联合优化,有效解决纹理重建过程中存在的几何、位姿误差以及各种噪声。在工作一基础上,本文利用可微分渲染优化三维模型顶点位置,对于三维模型中纹理丰富区域进行质心细分,以增加顶点和面片数目增强几何细节,引导顶点位置矫正。紧接着,本文提出交替优化策略分别对几何、纹理、位姿进行优化获取全局最优解。经过实验验证,本文提出的算法可以获得高保真的几何和纹理模型。
%==============================
\section*{未来展望}
由于在VR/AR、动画、视频游戏等领域有着广泛的应用前景,构建具有高保真纹理和几何图形的真实世界3D对象和场景一直是重要的问题。现阶段本文算法在纹理优化领域取得了一定的研究成果,但是还有很多问题需要进行解决。\par
目前,纹理优化领域所使用的数据集均由相关领域论文作者提供,缺乏第三方标准的公共数据集。由于拍摄场景、设备以及模型重建质量等因素的差异,评估纹理优化算法的效果面临一定的挑战。因此,下一步可以构建一些标准数据集,以供其他研究人员使用。这些数据集应该经过充分的策划和设计,以保证其中包含多样化的场景、设备和模型重建质量等。这样可以更加客观地评估纹理优化算法的表现,并促进该领域的发展。\par

当前的纹理优化算法主要以传统方法为主,与深度学习的结合还不够密切。对于基于数据驱动的算法,还存在一些缺陷。首先,这些算法更适合于小型场景的纹理优化,对于大型场景的表现不如传统算法。其次,在缺乏纹理或者仅具有单一颜色值的场景中,这些算法的纹理优化效果也会显著下降。未来的纹理优化算法需要具备更强的鲁棒性,并应当充分利用神经网络的学习能力,以对材质和灯光等因素进行建模,更加贴近真实世界的情况。\par
纹理依赖于几何模型存在,理论上几何模型重建越精确,纹理优化结果越好。不幸的是由于重建算法本身的缺陷或者物理遮挡使得几何模型会发生残缺现象。这破坏了几何模型的拓扑结构,为纹理优化带来新的挑战。未来可以先对几何模型缺失的地方进行补全,为后续获取高质量纹理模型提供保障。