% !TeX root = ../Template.tex
% 总结
\summary
\chapter{总结与展望}

%==============================
\section{工作总结}
本文首先介绍了基于RGB-D相机三维重建与纹理优化的背景与意义,以及纹理映射技术目前存在的技术挑战。然后,介绍了三维重建与纹理映射的相关工作做了总结与回顾。接着对于纹理优化相关技术做了梳理与总结。并且针对目前纹理优化中纹理存在模糊伪影以及裂缝现象我们提出基于学习的纹理优化方案,即利用可微分渲染技术矫正相机位姿,借助像素重生成管线重新生成纹理图片。为了保证在几何模型重建误差较大的情况下仍能获得高保真的纹理,我们提出一个联合优化算法对RGB-D相机重建的相机位姿、几何模型、纹理细节进行联合优化,最终获得高质量的几何模型和高保真的纹理。综上所述,本文贡献点有两个。\par
提出基于可微分渲染的相机位姿优化与纹理合成算法,获得清晰纹理。首先利用渲染将初始纹理与三维模型渲染至目标视角,通过与真实图片做损失,获得期望梯度以更新相机位姿。然后重新初始化纹理图,再将纹理图设置为优化变量利用对抗生成网络重合成貌似真实的纹理。经过纹理重合成,我们可以得到全局一致,高保真纹理模型。\par
提出联合优化算法同时对RGB-D重建的相机位姿、几何细节和纹理细节进行联合优化,有效解决纹理重建过程中存在的几何、位姿误差以及各种噪声。在工作一基础上我们利用渲染优化三维模型顶点位置而且然后对于三维模型中纹理丰富区域进行质心细分,以增加顶点和面片数目对几何细节增强,引导顶点位置矫正。然后我们提出交替优化策略分别对几何、纹理、位姿进行优化获取全局最优解。算法最后可以获得高保真的几何和纹理模型。
%==============================
\section{未来展望}
由于在VR/AR、动画、视频游戏等领域有着广泛的应用前景,构建具有高保真纹理和几何图形的真实世界3D对象和场景一直是一个重要的问题。现阶段本文算法在纹理优化领域取得了一定的研究成果,但是还有很多问题需要进行解决。\par
目前纹理优化领域所使用数据集均为,论文作者自己拍摄,一致缺乏第三方标准的公共数据集。拍摄场景、设备、模型重建质量均有所差别,这给我们评估纹理优化算法带来挑战。下一步可以构建一些标准数据集供他人使用。\par
纹理优化算法以传统为主,深度学习结合的很少。当前基于数据驱动的算法,存在较大缺陷。第一,只适合于小场景的纹理优化,对于大型场景的表现不如传统算法。第二,在无纹理或者颜色值单一的场景中纹理优化效果会显著下降。未来可以提出更加鲁棒性的算法,而且利用神经网络强大学习能力,对于材质,灯光分别进行建模以更加贴近真实世界。\par
纹理依赖于几何模型存在,理论上几何模型重建越精确,纹理优化结果越好。不幸的是由于重建算法本身的缺陷或者物理遮挡使得几何模型会发生残缺现象。这破坏了几何模型的拓扑结构,这为纹理优化带来新的挑战。未来可以先对几何模型缺失的地方进行补全,从而为后续获取高质量纹理模型提供保障。