% buaa基于ctexbook模板
% 模板选项:
%======================
% I.论文类型(thesis)
%--------------------
% a.学术硕士论文(master)[缺省值]
% b.专业硕士论文(professional)
% c.博士论文(doctor)
%--------------------
% II.密级(permission)
%--------------------
% a.公开(public)[缺省值]
% b.内部(privacy)
% c.秘密(secret=secret3)
% c.1.秘密3年(secret3)
% c.2.秘密5年(secret5)
% c.3.秘密10年(secret10)
% c.4.秘密永久(secret*)
% d.机密(classified=classified5)
% d.1.机密3年(classified3)
% d.2.机密5年(classified5)
% d.3.机密10年(classified10)
% d.4.机密永久(classified*)
% e.绝密(topsecret=topsecret10)
% e.1.绝密3年(topsecret3)
% e.2.绝密5年(topsecret5)
% e.3.绝密10年(topsecret10)
% e.4.绝密永久(topsecret*)
%--------------------
% III.打印设置(printtype)
%--------------------
% a.单面打印(oneside)[缺省值]
% b.双面打印(twoside)
%--------------------
% IV.系统类型(ostype)
%--------------------
% a.win(oneside)[缺省值]
% b.linux (linux)
% c.mac (mac)
%--------------------
% V.ctexbook设置选项(<ctexbookoptions>)
%--------------------
% ...
%======================
% 其他说明:
% 1. Mac系统请使用mac选项,并使用XeLaTeX编译。
% 2. 可加入额外ctexbook文档类的选项,其将会被传递给ctexbook。
%    例如:\documentclass[fontset=founder]{buaa}
% 3. CTeX在Linux下默认使用Fandol字体,为避免某些生僻字无法显示,在系统已安装方正
%    字体的前提下可通过fontset=founder选项常用方正字体。
%=================================================================
% buaa模板已内嵌以下LaTeX工具包:
%--------------------
% ifthen, etoolbox, titletoc, remreset,
% geometry, fancyhdr, setspace,
% float, graphicx, subfigure, epstopdf,
% array, enumitem,
% booktabs, longtable, multirow, caption,
% listings, algorithm2e, amsmath, amsthm,
% hyperref, pifont, color, soul,
% ---
% For Win: times
% For Lin: newtxtext, newtxmath
% For Mac: times, fontspec
%--------------------
% 请在此处添加额外工具包>>
%=================================================================
% buaa模板已内嵌以下LaTeX宏:
%--------------------
% \highlight{text} % 黄色高亮
%--------------------
% 请在此处添加自定义宏>>
%%=================================================================


\documentclass[master,public,oneside,win,AutoFakeBold]{buaa}


%=================================================================
% 开启/关闭引用编号颜色:参考文献,公式,图,表,算法 等……
%\refcolor{on}   % 开启: on[默认]; 关闭: off;

% 摘要和正文从右侧页开始
\beginright{on} % 开启: on[默认]; 关闭: off;

%盲评模式,开启后不显示学校等信息 ,注意自行修改致谢和学术成果
\blindmode{on}   % 开启: on; 关闭: off[默认];

% 空白页留字
\emptypagewords{[ -- This page is a preset empty page -- ]}

% 不显示超链接方框
%\hypersetup{hidelinks}

%=================================================================
% 论文题目及副标题-{中文}{英文}
\Title{基于可微分渲染的三维重建与纹理优化}{3D Reconstruction and Texture Optimization Based on Differentiable Rendering}
%\Subtitle{学位论文~\LaTeX{}模板}

% 学位级别
\Branch{工学硕士}

% 院系,专业及研究方向
\Department{计算机科学与技术学院}
% 一级学科/学科专业
\Major{计算机科学与技术}
% 二级学科
\Majorsec{计算机科学与技术}
%研究方向
\Field{纹理优化}

% 导师信息-{中文名}{英文名}{职称}
\Tutor{赵海峰}{Zhao Haifeng}{教授}
%\Cotutor{付燕平}{Fu Yanping}{教师}

% 学生姓名-{中文名}{英文名}
\Author{刘勇鑫}{Liu Yongxin}
% 论文编号
\StudentID{10357B00000000}

% 中图分类号
\CLC{AB00}

% 时间节点-{月}{日}{年}
%入学时间
\DateEnroll{09}{01}{2020}
%毕业时间
\DateGraduate{07}{01}{2023}
%论文提交日期
\DateSubmit{03}{01}{2023}
%论文答辩日期
\DateDefence{05}{21}{2023}

%%=================================================================
% 摘要-{中文}{英文}
\Abstract{%
	三维重建是VR/AR、动画游戏等领域的重要应用之一,需要实现高保真的纹理映射和精细的几何重建。然而,现有的三维重建算法在重建过程中会遭受各种噪声的影响,导致重建模型细节上的缺失,无法直接应用到以上场景中。三维重建结果与实际需求不匹配,受以下几个因素的的影响和破坏:(1)深度相机测量物体深度时,易受到运动模糊、相机抖动、光源干扰等因素影响,导致出现度量误差。(2)相机位姿估计时,估计误差会不断累计并传递至下一帧,即使有回环检测和回环矫正,也不能保证相机位姿估计完全正确。(3)三维重建算法通常采用截断符号距离场来表示重建模型,并应用加权平均策略来降低数据噪声。这种加权平均策略会在一定程度上降低几何细节的复杂性,从而使得重建模型表面变得平滑,失去模型本来的高频几何细节。这往往导致重建的三维模型与真实场景在几何形状上存在误差。\par
	针对三维重建过程中存在的挑战,本文基于可微分渲染提出了一个端到端的优化方法,对相机姿态和纹理进行改进。首先,使用光度一致性纠正几何模型顶点到每个视角的映射关系,然后利用对抗神经网络学习误差容忍度量,容忍纹理重建过程中存在的各种误差。本文方法经过实验证明能够消除初始纹理存在的缺陷,如模糊、伪影和裂缝等。相比于传统方法,本文方法适用于各种具有挑战性的场景,并且可以弥补传统方法泛化能力不足的缺陷。\par
	虽然矫正相机位姿与纹理重合成算法可以降低纹理映射中的位姿误差和噪声影响,但是当重建模型存在严重错位或几何细节缺失时,难以达到理想效果。因此,本文提出了基于可微分渲染的几何模型优化方法,它能够通过更新几何模型的顶点位置来重建模型的表面形状。此外,本文采用基于三角形质心的自适应细分方法,以增加几何模型表面的平滑度,从而更容易恢复高频几何细节。最后,本文基于可微分渲染提出了一个联合优化深度学习框架,对相机位姿、纹理和几何模型进行联合优化。这种联合优化方式能够克服单独优化某一因素难以达到全局最优解的限制,从而得到高质量的重建模型。实验结果表明,在公共数据集上,本文提出的联合优化框架相比于其他最新纹理优化方法,在恢复三维重建模型高质量几何细节和高保真纹理细节方面表现更好。
}
{%
	3D reconstruction is one of the important applications in VR/AR, animation and gaming industries, requiring high-fidelity texture mapping and fine geometric reconstruction. However, existing 3D reconstruction algorithms are often affected by various noise during the reconstruction process, leading to missing details in the reconstructed model that cannot be applied directly to the above-mentioned scenarios. Three-dimensional reconstruction results often do not match actual requirements due to the following factors: (1) when measuring the depth of objects using a depth camera, errors can easily occur due to factors such as motion blur, camera shake, and light source interference. (2) When estimating camera pose, estimation errors accumulate and propagate to the next frame, and even with loop closure detection and correction, camera pose estimation cannot be guaranteed to be entirely accurate. (3) 3D reconstruction algorithms usually adopt truncated signed distance fields to represent the reconstructed model, and apply weighted averaging strategies to reduce data noise. Such averaging strategies can reduce the complexity of geometric details to some extent, resulting in a smoother surface of the reconstructed model, losing the high-frequency geometric details of the model itself. This often leads to geometric errors between the reconstructed 3D model and the real scene.\par
    This thesis proposes an end-to-end optimization method based on differentiable rendering to address the challenges in the 3D reconstruction process, improving camera pose estimation and texture mapping. Firstly, photometric consistency is utilized to correct the vertex-to-view mapping in the geometric model. Secondly, an adversarial neural network is used to learn error tolerance metrics, allowing various errors to be tolerated during the texture reconstruction process. Experimental results show that our method can effectively eliminate initial texture defects such as blurriness, pseudoshadows, and cracks. Compared with traditional methods, our approach is applicable to challenging scenarios and can compensate for the lack of generality in traditional methods.\par
	Although the method of correcting camera pose and texture synthesis can reduce the pose error and noise in texture mapping, it is difficult to achieve the desired effect when the reconstructed model has serious misalignment or geometric detail missing. Therefore, this thesis proposes a geometry model optimization method based on differentiable rendering, which can rebuild the surface shape of the model by updating the vertex positions of the geometry model. In addition, this thesis adopts an adaptive subdivision method based on triangle centroid to increase the surface smoothness of the geometry model, making it easier to recover high-frequency geometric details. Finally, based on differentiable rendering, this thesis proposes a joint optimization deep learning framework to jointly optimize camera pose, texture and geometry model. This joint optimization method can overcome the limitation of optimizing a single factor and difficult to achieve global optimal solution, thus obtaining high-quality reconstructed models. Experimental results show that, on public datasets, the joint optimization framework proposed in this thesis performs better in restoring high-quality geometric details and high-fidelity texture details of 3D reconstruction models compared to other state-of-the-art texture optimization methods.
}
\vspace{-5em}

% 关键字-{中文}{英文}
\Keyword{%
三维重建,纹理映射,相机位姿估计,可微分渲染
}{%
	3D reconstruction,Texture mapping,Camera pose estimation,Differentiable rendering
}



% 图表目录
\Listfigtab{on} % 启用: on[默认]; 关闭: off;


% 缩写定义 按tabular环境或其他列表环境编写
%\Abbreviations{ \centering
%\begin{tabular}{cl}
%  $E$ & 能量 \\
%  $m$ & 质量 \\
%  $c$ & 光速 \\
%  $P$ & 概率 \\
%  $T$ & 时间 \\
%  $v$ & 速度 \\
%\end{tabular}
%}

% \PassOptionsToPackage{gbnamefmt=lowercase}{biblatex}
% \usepackage[backend=biber,style=numeric,sorting=none,gbpub=false,bibstyle=gb7714-2015,citestyle=gb7714-2015]{biblatex}
% \addbibresource[location=local]{ref.bib}


\begin{document}

%%=================================================================
% 标题级别
%--------------------
% \chapter{第一章}
% \section{1.1 小节}
% \subsection{1.1.1 条}
% \subsubsection{1.1.1.1}
% \paragraph{1.1.1.1.1}
% \subparagraph{1.1.1.1.1.1}
%--------------------
%%=================================================================

% 绪论
% !TeX root = ../Template.tex
% [绪论]
% 此处为本LaTeX模板的简介
\chapter{绪论}
计算机视觉是指利用计算机自动处理图像数据,以解决计算机无法通过人眼看到的图像问题的技术。
三维重建是指通过计算机视觉技术和图像处理技术,对真实世界中的物体进行三维建模的过程。三维重建技术可以帮助我们获取物体的准确三维信息,它可以帮助我们更好地理解和分析真实世界的物体,并且可以为计算机图形学、机器人、虚拟现实等应用提供基础。例如,在计算机图形学中,三维重建技术可以用来创建真实世界中的场景;在机器人领域,三维重建技术可以用来帮助机器人更好地感知环境;在虚拟现实领域,三维重建技术可以用来创建虚拟场景。RGB-D深度相机出现,可以同时捕捉图像和深度信息,为计算机视觉、机器人技术、虚拟现实等应用提供强大的数据支持。借助于RGB-D相机可以方便进行三维重建和纹理重建,极大的减少重建成本,恢复出更精确的三维场景的结构和纹理信息。

%%============================
\section{研究背景与意义}
从图片和视频中恢复出三维场景是计算机视觉领域重要目标之一,也是三维重建领域的核心研究内容,场景理解与分割的重要研究基础。三维重建有很多应用前景,例如在建筑、汽车制造、医疗、无人机、游戏、电影制作等领域都有广泛的应用。例如,在建筑领域,三维重建可以用来制作建筑物的三维模型,方便设计和建造;在医疗领域,可以用来制作身体的三维模型,帮助医生进行手术规划;在无人机领域,可以用来制作地面的三维模型,帮助无人机更好的导航;在电影制作领域,可以用来制作电影中的三维场景,更好的让观众感受到故事的真实感。\par
单纯的RGB图片无法正确恢复出场景的三维结构,因为RGB图片具有二义性,无法判断图片中的物体的远近关系,需要深度数据的参与。早期,三维重建没有直接的深度数据可得,需要利用多视角之间的视差关系,估计出深度图。受深度数据的影响,三维重建时间长,精度低,重建出的场景无法让人满意。得益于深度相机的出现,为我们重建出接近于真实场景的三维模型提供了可能。由于深度相机可以直接获取从相机光心到物体表面每个点的距离,所以基于RGB-D相机的三维重建技术不仅能达到实时性而且能获取场景精细结构。\par
虽然RGB-D相机可以获取准确的深度数据,从而提高重建模型质量。但是深度相机分辨率较低,容易丢失微小细节。不仅如此,深度相机采用红外光测量距离,容易受太阳光和室内其他光源的影响,产生深度测量噪声。并且相机本身在制造过程中可能会出现镜头变形,从而拍摄图片出现桶形畸变和枕形畸变使得三维模型顶点投影至平面时发生错位现象。以上深度相机固有的缺陷使得测量数据含有噪声,从而造成三维重建模型的瑕疵和高频细节的缺失。除此之外,三维重建时往往采用增量式融合方法,需要把每一帧的深度数据融合到全局模型中,而且每一帧的相机位姿估计依赖于前一帧,由于估计误差不可避免,这种误差一直会累计并传递至下一帧,发生相机位姿漂移现象。即使有回环检测等位姿修正方法,但是位姿误差仍然无法消除,造成重建出的三维模型顶点偏离预期位置。再者,生成纹理贴图时,三维模型中每个顶点会投影至每个彩色图片上收集颜色。由于几何误差和相机位姿估计误差的存在,每个顶点投影至不同视角的RGB图片时会得到不同的颜色,造成纹理映射出现模糊伪影现象。\par
由于存在以上误差,纹理映射结果无法让人满意,还需要人工后处理才能应用至其他领域。要想获取清晰纹理,必须首要解决相机位姿估计不准确的问题,再者解决由于几何误差导致纹理不能和几何模型对齐的影响。先前的工作在提升纹理映射质量方面做出了许多努力。一些方法致力于提升重建模型质量,间接提高纹理映射结果,一些方法采用扭曲方法弥补相机位姿的误差,还有方法采用联合优化方法,矫正相机位姿、恢复三维重建模型几何细节,估计场景光源,共同提升几何和纹理质量。受可微分渲染方法在单视图三维重建领域的成功,本文方法首先采用可微分渲染方法优化相机位姿,再将可微分渲然模块嵌入对抗神经网络中,合成貌似真实的纹理图。除此之外,本文基于可微分渲染又提出联合优化框架,不仅能提升纹理质量而且能提升几何模型的高频细节,从而得到高保真的纹理。\par
综上所述,本文提出利用可微分渲染优化相机位姿和用对抗神经网络合成纹理策略解决纹理映射中存在的模糊伪影问题。在第一个框架基础上提出联合优化方法,提升几何模型重建质量,弥补纹理与几何模型存在的错位现象,从而获得高保真的纹理。
%%============================
\section{研究现状}
\subsection{三维重建}
三维重建技术错综复杂,随着视觉传感器的进步也不断催生出各种三维重建算法。基于RGB-D相机的三维重建技术来源于SLAM(Simultaneous Localization and Mapping),是计算机视觉中的一种技术,它可以通过观察周围的环境来确定机器人的位置和场景的地图。根据传感器的不同可以分为视觉SLAM、激光雷达SLAM、RGB-DSLAM等。这里为了方便,RGBD相机出现以前的SLAM技术,统称为传统SLAM。相比于传统SLAM,基于RGB-D相机的三维重建既采用深度信息也采用彩色信息,并且深度信息为主体,而传统的SLAM主要用彩色信息。其次侧重点也不相同,传统SLAM以定位为主体,建图次之,而且以稀疏图为主。基于RGB-D相机的三维重建以建图为主体,构建稠密高质量的地图以满足实际应用需求。自KinectFusion~\cite{RichardNewcombe2011KinectFusionRD}利用RGB-D相机实时重建出精密地图,效果惊艳,引发了RGB-D SLAM的潮流。所以后续大部分重建作品为了保持一致,名称中均带有fusion,所以基于RGB-D相机的三维重建与基于Fusion的三维重建的说法等价。基于RGB-D相机的三维重建可以细分为静态/动态场景的重建,但是本课题纹理优化主要基于静态场景,所以本文只叙述静态重建相关工作。\par
2011年以KinectFusion\cite{RichardNewcombe2011KinectFusionRD}为典型代表的重建问世
,使用Kinect传感器采集的深度数据实时融入到TSDF(Truncated Signed Distance Function)表示的模型中。TSDF是通过构建体素空间重建三维实体,并且体素空间分为多个小块,每个体素小块存储该小块与其最近的物体表面的距离。KinectFusion使用帧到模型的迭代最近点算法跟踪当前帧与全局模型的对应关系,获取每帧图像的相机位姿变化。但是由于采用稠密体积表示,消耗内存,并且易受相机位姿误差影响,所以很难重建大型场景。Thomas Whelan等人扩充了KinectFusion系统,提出了较为完备的Kintinuous~\cite{ThomasWhelan2012KintinuousSE}。它融合了回环检测和回环优化,以抵抗位姿估计中的噪声。并且使用变形图做非刚性变换,更新点的坐标,适合大场景的三维重建。进一步地,Thomas Whelan又提出ElasticFusion~\cite{ThomasWhelan2015ElasticFusionDS},采用面元(Surfel)的表示方法,用于小型场景的重建。它将模型到模型的局部回环和全局回环检测结合重建地图,并通过优化地图方式保证重建结果的全局一致性。Prisacariu等人提出InfiniTAM v3~\cite{VictorAdrianPrisacariu2017InfiniTAMVA},跨平台实时的大范围深度信息融合与跟踪框架,利用哈希表存储隐式体积表示,因而能重建比KinecFusion更大范围的3D环境。ANGELA DAI等人提出并行化的优化框架Bundlefusion~\cite{AngelaDai2016BundleFusionRG}利用光束平差法(BA)来优化三维重建模型中相机的位置和物体的三维坐标,使得重建结果与观测数据之间的误差最小。充分利用基于稀疏特征以及稠密几何和光度匹配提出的对应关系,合并多个观测模型提高三维重建结果的精度。最近,神经辐射场在三维重建领域大放异彩,Dejan等人~\cite{DejanAzinovic2021NeuralRS}提出基于神经辐射场的非实时三维重建框架,利用神经网络存储符号距离场(SDF)隐式表达重建模型,可产生比单独使用基于颜色或深度数据的方法更详细和完整的重建结果。\par
虽然以上方法采用各种方法试图抵抗重建过程中采集的深度数据噪声以及估计相机位姿的累计误差,以及不同表面之间固有的缺陷使得生成的三维模型仍然存在几何细节丢失,瑕疵等现象。

\subsection{纹理优化}
在三维重建中,除了重建出物体和场景的高质量的几何模型之外,人们往往更加 专注于场景与物体的外观,也就是场景和物体的材质和纹理颜色。要恢复三维重建模 型的纹理颜色就需要我们对重建的模型进行纹理映射。好的纹理映射结果可以弥补三 维重建中几何模型的缺陷,并生成逼真的纹理映射的结果。在一些应用中,重建模型 的纹理映射的要求比重建高质量的几何模型的要求更高,这就需要重建出更高质量的 纹理结果。特别是在一些虚拟现实、增强现实、虚拟试衣以及数字游戏等应用中。认 识到纹理映射的重要性之后,近几年三维重建工作者在三维重建纹理映射方面进行了 大量地研究,并取得了一些显著的进展。
三维重建旨在利用计算机技术构建真实场景,除了重建出几何模型,外观模型也是必要的,而且人们往往更加关注三维物体的外观。高质量的纹理映射不仅能弥补几何重建中的缺陷而且能进行纹理编辑从而实现外观自由控制。但是重建高保真的纹理映射比单纯的几何重建更加具有挑战性,因为人眼往往对外观存在的伪影模糊裂缝等现象更加敏感。 \par
为了恢复三维重建模型高保真的纹理细节,近十几年,专家学者在三维重建领域做出了许多努力,在实现三维重建与纹理映射方面取得的显著的进展。本文从影响纹理优优化结果的不同因素:不精确的相机位姿、几何模型和纹理出发,按不同的优化思路把相关的算法分为一下几类。\par
\vspace*{2mm}\noindent{\bf 基于面投影的方法:}它为每个面片选择最佳视角Lempitsky等人~\cite{lempitsky2007seamless}使用成对儿的马尔可夫随机场为几何模型生成纹理图集。其中数据项为每个面片寻找最佳视图,平滑项最小化纹理块之间的细缝,通过图割和$alpha$膨胀~\cite{boykov2001fast}最小化能量函数。这种方法面临一个具有挑战性的问题,即如何减轻相邻纹理之间的视觉接缝:由于相机参数或重建的几何形状略微不准确,纹理补丁可能会在边界处不对齐,产生重影,并导致强烈可见的接缝。Lempitsky等人提出全局颜色矫正方案,调整将不同块边界顶点颜色,以使得相邻块之间的颜色充分接近。Waechter等人~\cite{waechter2014let}改进了一种全局色彩调整算法,不仅考虑相邻顶点之间的颜色差异额外考虑相邻边的颜色得到更为鲁棒的结果,在全局颜色调整后,使用泊松编辑调整目标图像区域的边界像素颜色,进一步减少细缝。Fu等人~\cite{fu2018texture} 提出了一种全局到局部的非刚性优化方法来调整摄像机的姿态漂移,利用重投影误差优化相机外参,并提出局部扭曲纹理坐标方法,纠正几何误差引起的纹理坐标漂移。基于面投影地方法,能得到清晰的全局纹理,但是由于相机曝光、位姿、几何误差等影响因素无法完全消除,因而会产生未对齐地局部纹理。\par
\vspace*{2mm}\noindent{\bf 基于顶点的方法:}这种方法为三维模型顶点赋予颜色。将三位模型中每个顶点投影至可见的彩色图像上得到颜色,然后利用加权平均算法计算出该顶点的最终颜色,重复此步骤直到生成所有顶点的纹理。这种算法对相机位姿误差十分敏感,因为基于光度一致性假设顶点投影至不同视角会得到相同颜色,但是由于重建模型中各种误差,使得投影顶点会得到不同颜色,加权平均后发生颜色模糊现象。Zhou等人~\cite{zhou2014color}设计了一个纹理映射框架,通过刚性形变优化相机位姿,进一步通过局部图像的非刚性变形来进一步弥补相机位姿估计误差和几何重建误差。但该方法需要对网格模型进行细分,这将大大增加数据量,限制了通用性。\par
\vspace*{2mm}\noindent{\bf 基于块合成的方法:}基于块的方法来源于图像编辑~\cite{Barnes:2009:PAR},借助于代替纹理优化方案,基于块合成方案通过为每个源图像合成一个对齐的图像,我们纠正了由几何、相机姿势和输入图像的光学畸变引起的不对准。Bi等人~\cite{bi2017patch}采用基于 patch 的图像合成方法来生成一个新的完全对齐的目标纹理图像 来消除纹理图像之间的不对齐,从而避免纹理结果的模糊和重影。最近fu等人~\cite{fu2021seamless}提出新的纹理映射方法,使用一个三向相似度函数来重新合成纹理图边界条纹内的图像上下文,减少纹理接缝的出现。最后引入全局颜色协调方法来解决从不同视点捕获的纹理图像之间的颜色不一致,生成视觉逼真的纹理映射结果。基于块的合成方案不仅用与纹理块合成而且用于纹理补洞等其他应用。\par
\vspace*{2mm}\noindent{\bf 基于联合优化的方法:}联合优化方案对重建过程中各种误差,分别进行优化,并使用交替循环策略进行联合优化。Robert Maier等人~\cite{RobertMaier2017Intrinsic3DH3}提出了一种基于SFS(shape-from-shading)和空间变化的球谐光照函数的子体优化方法,同时优化几何、纹理、相机姿态和场景照明。可以获得纹理一致的高质量三维重建。但该方法依赖于SFS,需要分解场景的光照,容易导致纹理拷贝问题。最近工作中Fu等人~\cite{YanpingFu2020JointTA}提出根据颜色和几何一致性以及高频法线线索对重建网格进行优化,有效地克服了SFS产生的纹理拷贝问题,从而得到了更加高质量的重建结果。这种方法相比之前只优化纹理方法,在几何模型重建细节缺失较大时,仍然能产生清晰纹理。基于此本文第二个工作也提出联合优化框架实现更加鲁棒的结果。\par
\vspace*{2mm}\noindent{\bf 基于深度学习的方法:}对抗生成网络~\cite{NIPS2014_5ca3e9b1}强大拟合能力在图像翻译领域大放异彩,合成与真实世界充分接近的彩色图片,例如优秀开源框架pix2pix~\cite{isola2017image}cycleGAN~\cite{  zhu2017unpaired}等。受基于块合成纹理方法启发,Huang等人~\cite{JingweiHuang2020AdversarialTO}使用从弱监督视图中获得的条件对抗损失为近似表面生成逼真的纹理,使用基于学习的方法训练纹理目标函数,以保持对摄像机姿态和几何畸变的鲁棒性。与我们相似的工作zhang等人~\cite{9705143}借助于可微分渲染方法提出了一种联合优化方法,将几何、纹理和相机姿态共同纳入一个统一的优化框架中,并采用一种自适应交织策略,提高优化的稳定性和效率。与我们的方法类似但是我们的方法在优化几何时采用自适应性细分,从而当几何误差过大时,我们的方法仍能恢复出到几何模型的细节,并能获取高保真的纹理。
\subsection{可微分渲染}
可微分渲染是通过梯度设计,能够将渲染输出的梯度方向传播至三维实体,从而弥补了二维和三维之间的差距,同时可以允许神经网路在操纵渲染图像的同时优化三维实体,无需额外的三维标注。基于不同的三维实体如体素、点云、SDF、网格有不同的可微分渲然方法,因为网格能够表达三维模型的拓扑结构,又无需关注三维实体内部的构成,这种表示方式不仅灵活而且节省内存,所以本文只关注基于网格的可微分渲染。
Loper和Black~\cite{MatthewLoper2014OpenDRAA}设计了基于网格的通用框架OpenDR,近似可微渲染器。Kato等人~\cite{MatthiasNiener2013Realtime3R}提出了一种神经3D网格渲染器,用手工设计的函数来近似光栅化后向梯度。SoftRas~\cite{ShichenLiu2019SoftRA}以概率方式将每个像素分配给网格的所有面,保证前向后向操作均可微。pytorch3d~\cite{ravi2020pytorch3d}基于SoftRas方法,设计出通用的基于网格的可微分框架,不仅方便地更新几何模型而且还能操作纹理贴图。我们的方法基于pytorch3d,借助于可微分渲染框架,将梯度反向传播并更新场景中相机位姿,几何和纹理贴图。最近fuji等人依据DIB-R~\cite{chen2019_dibr}发布基于网格的可微分框架Kaolin~\cite{KaolinLibrary}能额外对材质进行编辑。为了聚焦于获取高保真纹理,在本文中除了几何模型与纹理图集,并未考虑场景中其他参数,例如材质,灯光等。更多关于可微分渲染的详细介绍请参阅综述\cite{HiroharuKato2020DifferentiableRA}。
%%----------------------
\section{本文的工作与安排}

\subsection{本文工作}
为了解决目前三维重建纹理映射存在泛化能力差、存在瑕疵等问题,本文工作围绕基于RGB-D相机的纹理映射主题,从三维重建模型的纹理优化与联合优化两个方面做了系统性的研究。\par
生成对抗网络可以学习误差容忍度量,能够对几何、相机位姿、灯光等各项误差容忍,并生成貌似真实的纹理,但是在相机误差过大时,生成的纹理结果仍然存在模糊伪影等现象,因此借助于可微分渲染我们提出优化相机位姿的方法,将模型顶点投影至所有可见的视角得到渲染图片,并于真实采集的数据进行比对产生期望梯度以更新相机位姿。然后使用优化后的相机位姿再用GAN网络~\cite{chanmonteiro2020pi-GAN}重新合成纹理图集。本方法在公共数据集上表现优异,相比于传统方法泛化性更强,而且对于误差较大场景鲁棒。\par
受fu等人联合优化方法启发,我们借助于可微分渲染方法我们优化几何模型顶点位置,为了恢复重建几何模型的高频细节,本文又提出自适应性细分方法,增加顶点面片数目以增强模型重建的合理性和精确性。除此之外,本文提出交替优化策略,首先利用重投影误差优化相机位姿,再经过模型细分后优化顶点位置,显著减少初始模型存在的误差,平滑模型参数中大部分噪声。最后再借助像素生成管线,对于剩余误差进行优化,从而得到高保真的纹理。实验证明我们的结果在各种场景中均能恢复出高保真的纹理,符合真实世界的三维模型。。
\subsection{结构安排}


本文的具体结构如下: \par
第一章 绪论。首先介绍基于RGB-D相机的三维重建与纹理映射背景和研究意义,其次,依照时间顺序
探讨与本文相关的领域:基于fuison的三维重建、纹理映射、基于网格的可微分渲染的国内外研究现状最后介绍本文的研究思路和过程,以及论文的结构安排。\par
第二章 纹理优化相关技术介绍。在本章中,首先介绍数据获取设备,重建场景的表面表示,纹理表示与获取以及可微分渲染的原理。详细描述重建模型的各个组件如何影响纹理映射结果,以及各个部分与纹理映射的关系。\par

第三章 基于可微分渲染然的纹理优化。本章首先利用可微分渲染技术优化相机位姿,再借助对抗生成网络生成纹理图像,并在已有的数据集上验证方法结果。对应于本文第一个工作。\par

第四章 基于自适应细分的重建模型与纹理优化。详细介绍对于RGB-D三维重建模型的联合优化算法,共同优化相机位姿、几何模型与纹理图集。对应于本文第二个工作。\par

第五章 总结与展望。总结本文主要工作,介绍本文方法存在的缺点与不足,并对纹理优化方法未来的研究工作做出展望。



% 说明
% !TeX root = ../Template.tex
% 本LaTeX模板的一般使用说明
%\chapter{绪 论}
\setcounter{figure}{0}
\setcounter{table}{0}
\setcounter{algocf}{0}


\chapter{相关理论基础}
三维重建和纹理优化是一个结合计算机图形学、计算机视觉和代数等知识的领域。它包含多种复杂的理论。本文首先介绍了三维重建和纹理模型所使用的数据获取方式;接着,阐述了如何对重建的三维模型进行表示,以及各种表示方法的优缺点。然后,介绍了常见的纹理表示方式以及如何实现纹理与模型之间的映射,同时给出基于加权融合方法生成初始纹理的详细步骤。最后,本文描述了可微分渲染的基本原理,并介绍了可微分渲染技术在重建模型和纹理优化中的作用。
%
% 缺乏示例图/Kinect相机、小孔成像原理图
%
\section{数据获取}

深度相机是一种能够获取三维场景信息的相机。它通常使用激光或者红外干涉仪来测量距离,并且能够根据这些距离信息生成深度图像或者三维点云。本文实验数据集主要采用微软发布的 Kinect \emph{v}1 相机采集场景的深度图序列和彩色图序列,部分场景下用Kinect \emph{v}2深度相机采集。Kinect \emph{v}1采用结构光方法进行深度测量,通过投影仪发射伪随机散斑红外光点,然后收集被测物体采集结构光图像并计算出相机至物体表面距离。深度相机测量范围在0.5m $\sim$ 4.5m之间,并且随着距离的增加,测量精度有所下降。Kinect \emph{v}1相机分辨率只有640 $\times$ 480,彩色图像的低分辨率使其易于模糊。此外,相机对光源较为敏感,容易受到室内光源的干扰。由于相机硬件的不完美和这些干扰项的存在,采集的数据中可能含有噪声。因此,在三维重建中,通常需要在算法层面处理或纠正这些噪声对数据的影响。Kinect \emph{v}2 相机在第一代基础上做了改进,利用飞行时间(Time of Flight,TOF)原理测量物体深度值。该传感器向目标发射红外光线,通过接收反射光线并记录光脉冲的飞行时间,估算出物体深度。Kinect \emph{v}2 相机可以拍摄分辨率为 1920 $\times$ 1080 的高清彩色图像和深度值更准确的深度图像。由于彩色图像和深度图像的分辨率不同,因此在使用前需要对彩色图像的相机镜头和深度图像的相机镜头进行分别标定。得到采集的深度数据后,一般会将其存储在单通道无符号16位的\emph{png}图像中,单位为毫米。\par 
在深度图上,像素和三维模型上的顶点存在着一一对应关系。为了准确描述三维物体的姿态和位置,通常需要使用相机坐标系、世界坐标系、图像坐标系、像素坐标系和模型坐标等不同的坐标系来定位物体。根据小孔成像原理,三维物体最终会映射到图像平面上。在将模型投影到图像平面时,首先将位于世界坐标系内的模型顶点通过刚体变换转换到相机坐标系中,然后按照透视投影原理将顶点变换到图像坐标系的像平面上。其中,图像坐标系的物理尺度为米,坐标原点在图像中心。为了得到统一的尺度表示,不受传感器尺寸的影响,还需要将像平面上的点转化为像素坐标系,并通过畸变矫正处理得到最终的图像数据。不失一般性,设三维模型上一点$\boldsymbol{p} = (X_w,Y_w,Z_w)^\top $,定义刚体变换为$\mathbf{g}\left(\boldsymbol{p}, \mathcal{T}\right)=\mathcal{T} \boldsymbol{p} = \boldsymbol{R} \boldsymbol{p}+\boldsymbol{t}$ 。
其中,$\mathcal{T}=\left(\boldsymbol{R},\boldsymbol{t}\right) \in \mathrm{SE} (3), \boldsymbol{R}   \in \mathrm{SO}(3) \text { 和 } \boldsymbol{t} \in \mathbb{R}^{3}$。刚体变换后得到相机坐标系中一点$\boldsymbol{p'}=(X_c,Y_c,Z_c)^\top$。则$\mathrm{u}$为在图像平面$\mathrm{I}$上的投影为:
\begin{equation}
 \mathbf{u}\left(X_c, Y_c, Z_c\right)=\left(\frac{X_{c} f_{x}}{Z_{c}}+c_{x}, \frac{Y_{c} f_{y}}{Z_{c}}+c_{y}\right)^{\top}
\end{equation}
其中,$f_x$和$f_y$表示焦距长度,$\left(c_{x}, c_{y}\right)^{\top}$表示相机光心位置。最终得到二维像素位置$\boldsymbol{x}=(u, v)^{\top}$。设相机内参为$K$,则三维空间至图像平面的映射关系如下列公式所示:
\begin{equation}
\left(\begin{array}{c}
u \\
v \\
1
\end{array}\right)=\frac{1}{Z}\left(\begin{array}{ccc}
f_{x} & 0 & c_{x} \\
0 & f_{y} & c_{y} \\
0 & 0 & 1
\end{array}\right)\left[\begin{array}{cc}
\boldsymbol{R} & \boldsymbol{t} \\
\mathbf{0}^{T} & 1
\end{array}\right]\left(\begin{array}{c}
X_{w} \\
Y_{w} \\
Z_{w} \\
1
\end{array}\right)=\frac{1}{Z} \boldsymbol{K} \mathcal{T}\left(\begin{array}{c}
X_{w} \\
Y_{w} \\
Z_{w} \\
1
\end{array}\right)
\end{equation}

当相机透镜产生畸变时,它会影响成像平面上的像素位置和图像的形状,使得相邻像素之间的距离不再均匀。这种现象称为相机透镜畸变,与针孔模型的假设相悖。因此,如果在三维重建中使用了畸变图像,那么将会导致误差的积累,最终影响重建结果的精度和质量。为了避免这些问题,需要对相机透镜进行标定和矫正。相机标定是一种过程,旨在确定相机内部和外部参数,包括焦距、光心、透镜畸变等,这些参数可以用于后续的三维重建。透镜畸变校正是一种技术,可以消除或减少透镜畸变的影响。相机畸变可分为径向畸变和切向畸变。径向畸变是一种常见的畸变类型,光线在穿过透镜时在径向方向上发生弯曲,导致物体形状变形,尤其在图像边缘处更为明显。另一种类型是切向畸变,由于透镜材质和折射率等制造工艺缺陷导致光线在切向方向上发生弯曲,导致图像像素列倾斜或扭曲,通常在图像中心位置出现。对于径向畸变,可以通过泰勒级数进行矫正。而对于切向畸变,可以使用查找表或多项式方法进行校正。通过相机标定和畸变校正,可以提高三维重建的精度和可靠性,减少误差的积累,从而得到更加准确的重建结果。
\vspace{-2em}

\begin{equation}
\begin{array}{l}
x_{\text {corrected }}=x\left(1+\mathrm{k}_{1} \mathrm{r}^{2}+\mathrm{k}_{2} \mathrm{r}^{4}+\mathrm{k}_{3} \mathrm{r}^{6}\right) \\
y_{\text {corrected }}=y\left(1+\mathrm{k}_{1} \mathrm{r}^{2}+\mathrm{k}_{2} r^{4}+\mathrm{k}_{3} \mathrm{r}^{6}\right)
\end{array}
\end{equation}
其中,$x,y$为经过透视投影后的图像位置坐标,$r$为坐标点距成像中心的距离。切向畸变假设透镜和图像平面之间存在一个小的平移量,可以通过平移参数进行矫正。以下是切向畸变矫正公式。
\begin{equation}
\begin{array}{l}
x_{\text {corrected }}=x+\left[2 p_{1} y+p_{2}\left(r^{2}+2 x^{2}\right)\right] \\
y_{\text {corrected }}=y+\left[2 p_{2} x+p_{1}\left(r^{2}+2 y^{2}\right)\right]
\end{array}
\end{equation}

由上述公式共有$k_1,k_2,k_3,p_1,p_2$五个参数,求解参数过程就是图像去畸变过程,具体原理可参考相机标定原理,如经典的棋盘格标定\upcite{888718}。

%
% 缺乏示例图,TSDF、面元表示,网格模型图
%
\section{表面表示}
纹理是三维模型的视觉外观的具象化表示方式,它依赖于三维模型的存在。对于三维模型的表达方式,通常会根据构成元素的不同进行分类,具体选用哪种表面表示方式取决于重建算法和应用场景。常见的三维模型表示方式包括显式表示和隐式表示。显式表示方式包括点云、网格和体素等,而隐式表示方式包括占据网格和符号距离场等。选择不同的表面表示方式会影响到重建的效率、精度和所需存储空间等方面。\par

在SLAM重建过程中,为了满足实时重建要求,一般采用隐式表示方法中的截断符号距离场或面元素(Surfel)。为了建模真实场景,首先需要定义建模的三维空间。例如,KinectFusion使用物理尺度下大小为$7m \times 7m \times 7m$的立方体表示室内空间。然后将三维空间划分为$128 \times 128 \times 128$个立方体小块,每个小块作为一个体素。随着分辨率的提高,体素数量急剧增加,相应的计算代价和内存代价呈指数级提高。因此,普通的体素表示方法不适用于大场景物体的重建,更适用于小场景的重建\upcite{nguyen2018rendernet}。由于体素表示相对规整,很容易进行数据更新和融合,并且易于进行GPU并行化处理,因此在基于RGB-D的三维重建中,常使用体素的隐式表示法,例如TSDF(Truncated Signed Distance Function)。在TSDF模型中,每个体素小块存储相对于物体表面位置的距离值,正值表示在表面外,负值表示在表面内。由于重建过程中只关处于物体表面附近的体素,不需要考虑所有体素,因此计算效率大幅提高。\par


TSDF算法\upcite{BrianCurless1996AVM}可分为三个步骤:第一,需要对整个三维场景划分为$N$个规则的网格以表示整个场景。TSDF中每个网格存储距离值$D(p)$和相关权重$W(p)$,其中$p$为体素空间中的一个网格。第二,计算当前深度帧$\mathcal{Z}_{i}$上的TSDF值$d_i(p)$和权重$w_i(p)$。设$p$为单个体素在世界坐标系上的位置$p = (X_w,Y_w,Z_w)^\top$,根据上述介绍的投影流程获取在相机坐标系下z分量和投影至深度帧上的像素所代表的深度值后作差。则$d_i(p)$可表示为:
\begin{equation}
d_{i}(\boldsymbol{p})=\Psi\left(\left(\mathcal{T}_{i} \boldsymbol{p}\right)_{z}-\mathcal{Z}_{i}\left(\pi\left(\mathcal{T}_{i} \boldsymbol{p}\right)\right)\right) 
\end{equation}
其中,$\Psi(x) =\max \left(-1, \min \left(1, x\right)\right)$和投影函数$ \pi: \mathbb{R}^{3} \mapsto \mathbb{R}^{2}$。当前帧权重可表示为:
\begin{equation}
w_i(p) = \frac{ \cos \theta} { \left(\mathcal{T}_{i} p\right)_{z}-\mathcal{Z}_{i}\left(\pi\left(\mathcal{T}_{i}p\right)\right)} 
\end{equation}其中,$\theta$表示物体表面法向量与$p$到点光源连线的投影光线的夹角。
第三,基于当前帧所得的结果融入到全局模型结果中。当前帧的体素依赖于上一帧,整个体素的TSDF值为所有帧的权重乘以得到的TSDF值再平均。体素更新公式如下:
% \vspace{-1ex}
% \begin{align}
% \begin{array}{l}
% D_{i}(p)=\frac{W_{i-1}(p) D_{i-1}(p)+w_{i}(p) d_{i}(p)}{W_{i-1}(p)+w_{i}(p)}  \vspace{2ex}  \\
% W_{i}(p)=W_{i-1}(p)+w_{i}(p)
% \end{array}
% \end{align}


\begin{equation}
D_{i}(p)=\frac{W_{i-1}(p) D_{i-1}(p)+w_{i}(p) d_{i}(p)}{W_{i-1}(p)+w_{i}(p)}
\end{equation}
\vspace{-10ex}

\begin{equation}
W_{i}(p)=W_{i-1}(p)+w_{i}(p)
\end{equation}

\noindent 其中,$D_i(p)$表示融合第$i$帧后的体素加权融合结果,$W_i(p)$代表相关权重。融合所有深度帧后可以得到全局模型。之后可以用光线投射(ray casting)算法\upcite{511}或者移动立方体(Marching cubes,MC)算法\upcite{lorensen1987marching} 抽取出三维网格模型。
在基于RGB-D相机的静态三维重建中,ElasticFusion\upcite{ThomasWhelan2015ElasticFusionDS}使用面元素(Surfel)表示法。每个Surfel代表一个小的面片,包含面片位置、面片方向、颜色、权重、半径以及时间戳等信息。Surfel的大小和密度由表面曲率决定。相比于其他表示方法,Surfel表示法不需要考虑重建模型表面的拓扑结构,因此可以通过增加、删除和移动Surfel来修改表面。正是由于面元素表示方法的灵活性,ElasticFusion可以优化重建地图并提高重建和位姿估计的精度。\par

除了SLAM中的两种表面表示法,基于学习的三维重建中也使用点云、三维网格、体素的隐式表示方法进行非实时的重建。Lars Mescheder等人\upcite{LarsMescheder2018OccupancyNL}使用占据网格表示法,每个体素包含一个二进制占据状态(占据/未占据),隐含地将三维表面定义为分离两种状态的分类决策边界。理论上可以表达三维模型无限制的分辨率,且不需要过多的内存占用。有效解决了原始体素表示存在的高内存低分辨率的缺陷。相似地,以上介绍的SDF表示方式也很契合多层感知机(Multi-layer Perceptron,MLP)。它可以学习二维连续决策边界表示三维物体的等值面,这种表示不需要占用很多内存,也不需要很深的网络模型就可以表示高分辨率的三维场景。因此,NeRF\upcite{mildenhall2021nerf}、deepSDF\upcite{park2019deepsdf}等可以低成本代价创建出高质量的三维模型。此外,基于SDF表示法易于与光线追踪算法\upcite{hart1996sphere}结合。神经网络更容易找到决策边界即表面位置,提升重建效率,间接获得高保真的重建模型,如Deepvoxels\upcite{Deepvoxels}。点云作为原始的三维表面表示法,使用点集表达三维场景。好处是很容易利用传感器进行获取,并只需关注点集的三维坐标,不需要关注点之间的连接关系和表面拓扑结构。PointNet\upcite{qi2017pointnet} 和 PointNet++\upcite{qi2017pointnet++} 是基于点云的表达方法,被广泛应用于点云处理领域。它们使用最大池操作来提取全局形状特征,作为点生成网络的编码器,能够用于深度学习的分类和分割任务。PointNet++ 相比于 PointNet 具有更高的精度和更少的计算量。\par


然而,基于点云的表示法缺乏拓扑信息,生成三维网格变得不那么容易,往往使用泊松重建算法\upcite{kazhdan2006poisson}后处理生成三维网格,无法产生水密表面。基于三维网格表示,可以表达三维模型的表面拓扑结构,且无需关注三维空间内部,比较灵活,而且相比体素表示更节省内存。相比于点云,网格表示法不仅考虑点的三维空间坐标也关注点之间的连接关系,适用于产生水密表面。由于网格表示更容易进行存储、编辑,而且有较多的相关开源软件,MeshLab\upcite{LocalChapterEvents:ItalChap:ItalianChapConf2008:129-136}进行操作,因此基于网格的表示方法在实际应用中最广泛。但是基于网格表示的三维重建算法并不那么通用,因为大多数方法移动网格顶点或者重新生成网格面片时候会产生自相交的面片。而且只能生成简单拓扑结构的物体,并且需要预设置固定拓扑的模板网格来作为初始的重建模型。究其原因,网格在发生形变过程会改变顶点之间的连接关系,自然而然地造成拓扑结构的变化。此外,重建过程中无法准确预测顶点数量和连接关系,只能尝试多次网格细分增加面片数量。最后,网格模型的拓扑结构复杂多变,不像二维图像那样规则,也无法契合深度神经网络。因此,基于学习的重建方法只适用于重建拓扑结构简单且较小的物体。网格表面表示常用三角形或者四边形来表示面片,因为三角形具有稳定性,所以三角形面片最常见。\par
本文在重建过程中选择了网格表示方法,因为它更适合纹理图集,并且易于进行纹理编辑。SLAM重建完成后,本文用Marching cubes算法\upcite{lorensen1987marching}抽取三维网格。然而,在重建过程中,遮挡和噪声等因素会导致重建出的网格表面非常不完整,甚至不符合2D流形规则,这给纹理优化步骤带来了新的问题和挑战。

%
%缺乏示例图,各种纹理图
%
\section{纹理获取与表示}
\subsection{纹理表示}
纹理作为三维模型的外观表示与三维模型密不可分。为便于编辑,纹理通常以图像形式存储,并保留三维模型与纹理的映射关系。通俗地说,纹理是一种存储每个顶点颜色信息的方式。纹理不仅包含像素颜色,还维护了各个像素的位置与顶点之间的对应关系。使用纹理可以为三维模型增添有趣的外观,即使模型本身并不复杂,利用纹理可以伪造模型的几何细节,使其看上去更加丰富。例如,在游戏中常常使用纹理来展示丰富的场景,但计算代价并未增加。如MOBA游戏中的人物皮肤、FPS游戏中的枪械皮肤、墙壁贴图等。纹理通常以2D图片形式存在。根据纹理的表示方式和应用领域,纹理可以进一步分为一维纹理(只关注一个方向上的纹理);二维纹理(平面两个方向上的纹理);三维纹理(空间方向上的纹理)。相比于彩色图像,纹理反映了图像的本质特征,并不依赖于图像颜色或亮度变化,刻画了图形区域的像素灰度级空间分布属性。在纹理重建/优化中,通常使用以下表示方式。\par
\vspace*{2mm}\noindent{\bf 顶点纹理:} 纹理可以作为顶点颜色的表示,在存储三维模型时,最简单的方法是将对应顶点的颜色也直接存储在内部。这种表示方法非常适用于包含纹理网格细分的纹理优化算法,例如 Colormap\upcite{zhou2014color} 和 Intrinsic3D 算法\upcite{RobertMaier2017Intrinsic3DH3}。但是,为了保证图形的真实性,必须有足够多的顶点来指定足够多的颜色。这显然会增加存储开销,并且不易转换为其他纹理表示方式。\par
\vspace*{2mm}\noindent{\bf UV纹理:}这种表示方法使用彩色图像保存纹理,并使用\emph{mtl}文件存储所有纹理图像路径和材质信息。在纹理优化算法中经常使用此表示方法。相比于顶点纹理,它还可以额外指定材质参数,如环境光、高光、漫反射光系数、滤光透射率等。存储几何模型时无需额外存储顶点颜色信息,但会存储每个顶点的纹理坐标。以模型存储格式 \emph{obj} 为例,文件将存储顶点的空间坐标、顶点之间的连接关系以及每个顶点所对应的纹理坐标。假设模型上一点 $p$ 对应的纹理坐标是 $s = (u,v)$,其中 $u,v \in (0,1)$。假设纹理图片的大小分别是 $h$ 和 $w$,那么 $p$ 对应的像素位置为 $(h \times u, w \times v)$。由于在构建几何模型时没有考虑纹理坐标,因此只能通过特定的纹理图像生成算法来设定合适的纹理坐标位置。改变纹理图像必须相应的改变纹理坐标位置,这限制了应用范围。\par

\vspace*{2mm}\noindent{\bf 图集纹理:}这种表示方法比较特殊,来自于\upcite{ShichenLiu2019SoftRA}。相对于UV纹理,它不再将整个三维模型的纹理信息储存在一张图片中,而是预先定义多个正方形纹理块,让每个纹理块表示模型中特定的三角形面片纹理。这种表示方法更加适用于类似于shapenet\upcite{shapenet2015}数据集的场景,因为某些模型的部分面片可能没有纹理信息。此外,这种方法可以方便地和UV纹理进行转换,具有很好的灵活性。\par
\vspace*{2mm}\noindent{\bf 立方体纹理:}使用立方体的六个面来表示场景纹理。一种常见的纹理映射方法是使用球面UV映射,同时使用cubemaps\upcite{greene1986environment}来可视化球面领域。使用立体图来记录从一个单位球体投射出来的颜色信息,由一个单位盒子的六个面组成,因此被广泛用于图形学中的球形映射。另外一种标准的方法来可视化球形函数是使用等角图(equirectangle map)。相比于等角图,立方体地图失真较少,可以避免等角图在顶部和底部产生的失真区域。更多关于纹理表示的介绍可以参考\upcite{tarini2017rethinking,yuksel2019rethinking}\par
\subsection{纹理获取}
在估计了每帧彩色图像对应的相机位姿之后,可以使用图形学渲染管线,将三维模型的所有面片或顶点投影至所有彩色图上,并通过可见性测试获得每个面片或顶点在可见范围内的颜色信息。如果是面片,可以只投影至一张最清晰的彩色图像上,然后将结果存储在纹理图中,同时保留三维模型顶点与图像像素之间的映射关系;如果是顶点,可以通过加权平均或取置信度最高的投影位置像素的方式,融合所有投影位置的颜色值来得到该顶点的最终颜色信息。最后,可以将生成的纹理信息直接存储到以\emph{ply}格式的三维模型中。为了方便进行模型的投影,本文使用了已经预设好纹理坐标的网格模型。在本文的第一和第二个工作中,生成初始纹理图像的具体算法流程如下:\par

% \begin{enumerate}[label=(\arabic*)]
% \item ab
% \item cd
% \item efg
% \end{enumerate}



\begin{enumerate}[label=(\arabic*)]
    \item 预定义一张空白纹理图片$P$,$1024 \times 1024$大小。
    \item 设纹理坐标一点$U = (u,v,1)^\top$,将纹理坐标$u$利用相机外参$\mathcal{T}$与自定义内参$\mathcal{K}$投影至纹理图$P$上,得到像素点$X=(x,y)$。其中$X = U\mathcal{T}\mathcal{K}$。
    \item 记录覆盖像素点$X$的三角形面片$F$,记录面片索引$i$,计算像素$X$在图像平面上投影三角形的重心坐标$b_X\in(0,1)$坐标范围为$0 \sim 1$之间。
    \item 根据像素重心坐标$b_X$和面片索引$i$,计算出像素对应的顶点$v$和法线$\vec{n}$。
    \item 利用上一步求得的顶点$v$和面片$F$,向不同视角投影,求出顶点在视角$j$上投影点的像素颜色$c_j$,然后加权平均后获取最终颜色值$c= \frac{1}{N}  \sum_j^N c_j $,N为视角总数。
    \item 将所有顶点颜色存储到空白纹理图$P$上,将图片路径存储至\emph{mtl}文件中。
\end{enumerate}
重复上述步骤直到所有纹理坐标都投影完毕。

%
% 缺乏体素、网格、点云、隐式可微分渲染的优缺点介绍,需要扩充
%
\section{可微分渲染技术介绍}

渲染来自于计算机图形学,按照渲染方法可以细分为光栅化、光线追踪、体渲染等。光栅化渲染算法对每一个三角形面片进行单独渲染,可以在屏幕上快速显示3D图像,但是它不能处理高级的光照和阴影效果;光线追踪算法是一种基于物理光学原理的渲染技术,它可以模拟光线在3D空间中的传播、折射、反射等现象,实现高质量的图像渲染。在光线追踪算法中,光线从相机位置出发,经过每个像素,并沿着场景中的物体表面传播,最终到达光源或被吸收。通过对光线和物体的交互关系进行计算,算法可以计算出每个像素的颜色值和明暗程度,实现逼真的光影效果和高质量的全局光照效果。但是它需要较长的计算时间,常用于真实感图形渲染。体积渲染也是计算机图形学中的一种技术,用于在三维场景中模拟物体的材质和光线。它通过在每个像素上模拟光线与场景中的对象的交互来生成图像。与其它渲染技术不同,体积渲染可以模拟物体内部的材质和光线,并且可以生成逼真的阴影和渐变效果。常用的体积渲染技术有雾化、半透明渲染和烟雾效果等。由于以上算法渲染过程并不可微分,无法直接嵌入深度学习框架中。为了能用梯度下降算法更新场景参数,要么改变前向渲染过程\upcite{ShichenLiu2019SoftRA}使其可微分,要么手工设计后向传播的梯度\upcite{HiroharuKato2017Neural3M}。
神经渲染是一个前沿领域,它将机器学习技术与计算机图形学和物理学知识相结合,旨在通过训练神经网络来建模场景中几何、光照和材质之间的复杂关系,从而生成逼真的图像或视频。与传统渲染技术不同的是,神经渲染可以更灵活地生成视觉内容。通过利用可微分渲染方法,例如神经辐射场或可微分路径跟踪。利用神经网络来代替传统的渲染器,以获得可控和逼真的输出。目前在三维重建\upcite{yariv2020multiview}或者新视角合成\upcite{mildenhall2021nerf}\upcite{zhang2020nerf++}领域已经取得了显著效果。更多关于神经渲染的介绍请参阅论文\upcite{tewari2020state}。\par
可微分渲染(Differentiable Rendering,DR)旨在通过获得渲染过程的有用梯度来解决端到端优化的集成问题。通过微分渲染,DR弥合了2D和3D处理方法之间的差距,允许神经网络在操作2D投影的同时优化3D实体,而无需收集三维实体的属性或者标注。具体地,利用渲染器,相机内外参以及三维模型渲染出彩色图片,然后与真实采集的图片计算其差异,并且反向传播误差以更新场景的各个参数。例如,顶点位置、材质、场景灯光和相机位姿等属性。现有的可微分渲染研究按照渲染质量可以分为两类:基于物理的渲染方法,专注于生成逼真的图像质量或者追求更高的性能,往往使用渲染方程\upcite{kajiya1986rendering}的微分近似来模拟真实世界的光的反射与材质属性,渲染模型更加符合真实世界光照。第二种是基于经验的可微分渲染方法,使用简单的着色模型,如Blinn-Phong光照模型\upcite{blinn1977models}来对真实世界光照做经验上的近似。可微分渲染器通过从投影像素到3D参数生成导数来近似梯度,相比基于物理的渲染,参数量少而且效率高。\par
可微分渲染是一系列技术合集,并不具体指某项技术。根据三维实体的表面表示,它可以细分为基于体素表示、基于点云表示、基于网格表示或基于隐式表示的渲染方法。体素是三维空间中的单位立方体,其规则形状与三维卷积有天然的结合优势。Yang等人\upcite{nguyen2018rendernet}提出了一种基于3D卷积的可微分渲染卷积神经网络,它具有一种创新投影单元,可以将三维形状转化为二维图像。但由于体素的存储代价随着场景分辨率的提高而显著增高,因此重建场景大小受到限制。在基于学习的方法中,最常用的是体素隐式表示,其中占据网格和符号距离函数(SDF)等技术。如上一节所述,占据概率和截断符号可以隐式地表示为神经网络的决策边界,这样使用有限的内存代价就可以表达无限的分辨率。如\upcite{niemeyer2020differentiable,jiang2020sdfdiff,liu2020dist}等采用解析计算方法,使前向渲染过程可微分,从而获取高质量的渲染图像。近年来,在神经网络中以参数化的方式来表示几何信息越来越常见,这比三维显示表面表示更具有内存效率,可以充分利用GPU的并行处理能力。隐式表示的可微分渲染也逐渐成为主流。点云可以成功地集成到深度神经网络中,解决各种实际的三维问题,由于点云获取较为方便,因此成为数据表示的自然选择,在渲染时通过汇聚点云上所有可见的点特征,转换为彩色图像像素,如\upcite{wiles2020synsin,lassner2021pulsar}等。基于网格的渲染方法可以分为近似梯度、近似渲染和全局照明三类。通过改变前向或后向渲染的方式或者用梯度近似渲染三角形面片,可以将其渲染至图像中。但基于可微分渲染的推断几何模型形状方面,为了保持网格拓扑结构不发生变化,通常需要预先定义一个模板网格,限制了重建物体的范围。\par

本文利用可微分渲染来更新和优化纹理、几何模型形状和相机位姿,而并不关注材质、灯光等属性。在考虑性能和效率的情况下,本文使用基于光栅化的渲染器pytorch3d\upcite{ravi2020pytorch3d}。该渲染器包含两个组件,一个是光栅化组件:它选择影响每个像素的三角形面片,另一个是渲染组件:它计算每个像素的颜色。光栅化步骤:使用刚体变换和透视投影将三角形面片投影到图像平面上,得到覆盖每个像素的一组三角形面片。在传统的光栅化中,每个像素仅受其沿z轴最近的三角形面影响。在可微分渲染中,将为这些三角形面赋予不同的概率,其中每个面片的概率与其覆盖像素位置的顶点z坐标的相对位置相关,距离相机光心的距离越近,概率就越大。最后,按照概率加权融合不同的三角形面片,形成像素值,作为最终的像素颜色。渲染算法:Blinn-Phong光照模型中,物体颜色来源于三种光的累加:环境光$I_a$、漫反射光$I_d$和高光$I_s$。这些光照项会乘以各自的相关系数,最终得出像素的颜色。
\begin{equation}
    L = k_aI_a + k_dI_d+k_sI_s
\end{equation}
其中,$k_a,k_d,k_s$分别表示环境光照、漫反射光照和高光项系数。

\section{本章小结}
本节主要介绍了基于RGB-D相机的三维重建和纹理优化的相关知识,包括数据获取、表面表示、纹理获取和表示以及可微分渲染技术等。三维重建利用RGB-D相机采集的彩色图和深度图,本文首先介绍了深度相机的特点,接着阐述相机成像模型,并推导了三维模型顶点从世界坐标系向像素坐标系转换的过程。最后,介绍了图像去畸变的原理。第二部分中,本文首先介绍了三维重建模型的不同表面表示方式,讨论了各种表面表示方法在不同重建场景下的应用和优缺点。第三部分,本文介绍了纹理的特点以及与三维模型之间的映射关系,并列举了在纹理优化过程中使用的不同纹理表示方法。最后,简要介绍了本文为三维模型生成初始纹理模型所使用的算法流程。在最后一部分,本文首先介绍了三种常见渲染方法的特点及应用场景。其次,阐述可微分渲染的基本原理,并讨论了现阶段可微分渲染发展中的两个分支及其各自的特点。然后,本文具体介绍了可微分渲染技术的种类及其应用场景。最后,介绍了本文工作一和工作二所使用的可微分渲染框架。


% 示例

\chapter{基于可微分渲染的纹理优化}
\section{本章内容简介}

恢复出高保真的几何模型和纹理一直是三维重建的目标之一,相比几何模型在虚拟现实和游戏等真实领域通常对外观有更高要求。然而,由于数据获取与相机位姿存在各种误差使得重建模型与拍摄彩色图像无法完全对齐,如果不进行纹理优化那么直接生成的纹理会存在错位和缝隙。为了有效解决纹理映射存在的挑战,本文采用优化方法与合成方法共同生成纹理。首先利用可微分渲染生成彩色图片,利用重投影误差优化相机位姿,然后再加权平均所有顶点的投影颜色生成纹理图。单纯的优化相机无法完全对齐纹理与几何,所以本文提出基于对抗生成网络来重生成纹理以适配几何模型。本文在开源数据集上和我们自己拍摄的数据集上进行实验,实验结果表明我们的方法在各种挑战性的数据集上都能生成清晰的纹理。并且对比单纯的优化方法和纹理生成方法我们的方法具有优越性。
纹理,
\section{引言}
得益于消费机RGB-D相机的普及,人们能够很容易的对真实世界的场景进行建模得到场景和物体的三维模型。近几年,基于RGB-D的三维重建方法取得了前所未有的进步。然而,纹理优化/重建关注度不如几何重建的高,对于游戏、混合现实、虚拟现实等娱乐化数字产业来说纹理细节往往比几何细节更加重要。最近的工作已经可以获取高质量几何模型,但是纹理优化的质量无法令人满意。造成这一现象的根源在于重建模型的几何误差和相机位姿的估计误差,基于RGB-D的三维重建不可避免地会出现几何误差和相机漂移,这些误差会进一步造成最终图像和纹理不能完全与模型对齐,从而造成纹理出现模糊和重影等瑕疵。\par
要通过RGB-D相机获得高质量、高保真的三维重建模型,必须达到两个最基本的需求:物理意义正确并且高质量的几何模型和高保真度照片级别的纹理。然而,这两个基本需求很容易受到以下因素的影响和破坏:(1)深度相机获取的深度时很容易受到噪音、镜头扭曲和量化误差等因素的影响,造成深度测量出现度量误差。(2)在相机位姿估计时很容易出现相机累积误差造成相机位姿漂移。(3)现有的三维重建算法普遍采用TSDF(truncated signed distance field)数据融合技术来进行三维数据融合和表示,它虽然可以利用加权平均来消除大部分的噪音,但是它同时也会平滑掉很多高频的几何细节使得模型趋于平滑,而造成重建的模型出现几何误差(与真实几何模型之间的差异)。由于三维重建中几何误差和相机漂移,最终重建的纹理不可避免地会出现模糊和重影等瑕疵。这些因素导致重建的几何模型和纹理结果都无法满足当前一些应用的高质量几何和高保真纹理的需求。虽然三维重建技术有着广泛的应用前景,但是由于以上的这些因素很难重建出高保真的三维模型,所以重建的三维模型很难直接应用到其它的领域。\par

由于消费级的深度相机获得的深度图分辨率较低(Kinect第二代深度相机的深度图分辨率为512x424)、噪音大、视野小,测量距离有限。利用这样的深度数据进行重建不可避免的会出现几何细节的扭曲和丢失。此外,我们采用的TSDF 加权的策略对深度数据进行融合,这种加权平均的策略虽然可以过滤掉噪音,但是同时也会平滑掉一些高频几何细节,因此会进一步造成重建模型几何细节的丢失。对于纹理映射,由于相机位姿估计误差及几何误差的存在,不同视口得到的彩色图像不可能完全对齐,所以通过多视口加权生成的纹理不可避免的会出现模糊和重影。为了得到清晰的纹理,先前的工作使用不同的方法来克服重建模型中的误差。zhou等人~\cite{zhou2014color}采用非刚性抵抗相机位姿存在的误差。bi等人~\cite{bi2017patch}采用基于块合成方法,重生成纹理图以补偿几何模型和相机位姿中的误差。fu等人~\cite{fu2018texture}采用全局到局部优化方法以矫正投影矩阵以及纹理坐标。最近huang等人~\cite{JingweiHuang2020AdversarialTO}借助对抗神经网络,学习误差容忍度量,并使用像素重生成管线重新合成纹理图。\par
虽然神经网络具有强大的拟合能力,对多种不同误差都有很好的抵抗效果。但是当某些误差过大时,重生成的纹理仍旧存在模糊伪影现象。因此我们对相机位姿,几何模型分别进行优化后再借助神经网络重生成具有照片级清晰度的纹理图。\par
基于深度学习的方法在鲁棒性和易用性等方面相对于传统的纹理优化方面有优势,因为让神经网络学习如何抵抗模型重建过程的各种噪声,然后重新生成新的适配与几何模型的纹理。然而合成纹理的效果随着误差值的上升,效果逐渐下降,尤其是在相机位姿估计不准确的情况。某些数据中由于相机估计误差较大,对于几何模型顶点投影不同的彩色图像上,由于错位往往得到不同的颜色,在加权融合后纹理存在模糊伪影,而且单纯用对抗神经网络合成新的纹理图无法根除纹理优化现象。由传统方法fu等人的工作~\cite{fu2018texture}得到启发,我们将优化算法和纹理合成算法有机地结合起来。首先,借助于梯度下降算法,更新每个视角对应的相机位姿。然后,对每个顶点投影至可见视角中获取的颜色做加权平均算法生成纹理。纹理中存在模糊区域借助对抗生成网络重新合成。\par
传统的相机位姿优化算法往往借助李群和李代数~\cite{sola2018micro}理论优化相机位姿,计算代价高。受可微分渲染在三维单视图模型重建~\cite{liu2020general}~\cite{ShichenLiu2019SoftRA}中的成功启发,借助于可微分渲染我们可以对三维场景的各个组件进行优化,如顶点位置、灯光、相机投影矩阵等。借助于梯度下降算法进行值更新,并且很容易集成到神经网络中而无需额外参数,我们可以计算重投影误差以矫正相机位姿。经过实验证明,我们优化相机后生成的纹理与原始的纹理相比模糊伪影区域显著减少,接近于真实场景的外观。经过对抗生成网络后可以恢复出模型真实表面外观。经过公共数据集和我们自己拍摄的数据集的定量和定性结果比较,我们的算法比单纯用优化法和合成法,性能均有显著提高。
\section{相关工作}
近十几年纹理优化方法被进行广泛的研究,并提出了很多行之有效的算法。本文回顾和总结一些与我们工作相关的纹理优化方法。\par
\vspace*{2mm}\noindent{\bf 基于优化的方法:}生成纹理的方法有加权融合顶点颜色或者投影图像至几何模型,由于不可避免的误差因素存在加权融合算法会产生模糊伪影现象,投影方法图像会产生纹理缝隙。基于优化方法的常常会针对误差项进行优化矫正。zhou等人~\cite{zhou2014color}设计了一个新的纹理映射框架,该方法对相机位姿优化的同时对纹理图像进行 warping 操作来对几何误差造成的纹理不对齐进行校正。该方法需要对模型 进行细分,这样大大的影响了算法的效率。而且该方法的纹理结果很大程度上依赖于 网格细分的粒度。。3DLitle [48] 利用平面近似对整个场景的重建模型进行一个抽象的表示,然后他们使用平面 图元作为约束来优化相机位姿和纹理。最终他们可以得到一个平面抽象表示的带有高 保真度纹理的重建模型。然而这个方法使用平面对场景进行抽象的表示会丢掉很多几 何细节。Fu等人~\cite{fu2018texture} 提出了一种全局到局部的非刚性优化方法来调整摄像机的姿态漂移,并纠正几何误差引起的纹理坐标漂移。
\vspace*{2mm}\noindent{\bf 基于合成的方法:}最近,Bi 等人 [9] 采用基于 patch 的图像合成策略在每个视口重新 合成一张新的目标纹理图,这张新的纹理图相当于对原始纹理图进行了重组,使得所 有的生成的目标图之间能够完全对齐。该方法可以有效地解决由于相机位姿漂移和几 何误差造成的纹理图像之间不能完全对齐的问题。然而该方法的一个主要问题是它十 分耗时,效率太低。Huang等人~\cite{JingweiHuang2020AdversarialTO}使用从弱监督视图中获得的条件对抗损失为近似表面生成逼真的纹理,使用基于学习的方法训练纹理目标函数,以保持对摄像机姿态和几何畸变的鲁棒性。最近fu等人~\cite{fu2021seamless}提出新的纹理映射方法,使用一个三向相似度函数来重新合成纹理图边界条纹内的图像上下文,减少纹理接缝的出现。最后引入全局颜色协调方法来解决从不同视点捕获的纹理图像之间的颜色不一致,生成视觉逼真的纹理映射结果。Waechter 等人 [105] 在为每个面选择一个最优的纹理之后提出一个全局的颜色一 致性校正策略来减轻纹理之间的缝隙效应,该方法可以有效地减少多张纹理混合导致 纹理的模糊和重影,并有效地减轻纹理之间的缝隙。但是,该方法也没有进行任何相 机位姿和几何误差校正,所以并不能完全消除纹理之间的缝隙。


\section{算法流程}
我们的方法旨在通过RGB-D相机采集的深度图片和彩色图片,并将不同视角的彩色图适配到几何模型上以生成具高保真的纹理贴图。为了达成这一目标,我们提出结合相机位姿优化与纹理生成算法提高纹理模型的清晰度和保真度。该算法的具体流程分为四个步骤。
\noindent\textbf{模型重建:}RGB-D相机采集的原始数据为彩色图序列和深度图序列。在公共数据集上,一般会提供深度图集、彩色图集、对应的相机内外参以及用重建算法~\cite{SungjoonChoi2015RobustRO,AngelaDai2016BundleFusionRG,RichardNewcombe2011KinectFusionRD}重建出的初始几何模型。在我们自己拍摄的数据集上我们需要自己重建几何模型以及估计对应的相机位姿。我们利用重建算法~\cite{LongYang2018SurfaceRV}重建初始几何模型,并估计初始的相机位姿。\par
\noindent\textbf{预处理:}在预处理阶段,我们从所有帧中选出最清晰的彩色图像当作候选图像,消除冗余和重叠度较高的视角。同时保证所选关键帧的视角覆盖范围和原始集合一致。我们额外采用多视觉进行监督,避免单视角可能因相机飘移导致异常值影响纹理优化,我们为每一帧都选取邻近视角,同时用源视角和扭曲后的视角进行监督。\par
\noindent\textbf{相机位姿优化:}为了借用可微分渲染框架优化相机位姿,首先渲染三维模型至某一视角的彩色图像,然后再用真实图像与渲染图像做图像间的损失以优化相机投影矩阵。优化相机位姿完成后,我们将模型顶点投影至所有可见视角获取顶点颜色,再用加权平均算法融合颜色以生成纹理,为了便于后续纹理优化我们将颜色以UV纹理形式保存。\par
\noindent\textbf{纹理优化:}由图像翻译模型~\cite{isola2017image,wang2018high}得到启发,我们借助对抗生成网络在初始纹理基础上重生成新的UV纹理,我们已知纹理到图像的映射关系,并且获取相对准确的相机位姿。用源视角$I_A$的邻近视角$I_B$扭曲到源视角当作真例,渲染图像为假例,以对抗生成模型优化纹理图。\par

\subsection{数据处理}
\noindent\textbf{重建模型:}RGB-D数据我们使用消费级深度相机kinect V1或者kinect v2在固定曝光和白平衡模式下拍摄。拍摄完成后已知信息只有初始拍摄所得的为深度流和彩色流以及相机标定后的内参,并没有相机位姿与重建模型。由最近纹理优化方法G2Tex~\cite{fu2018texture}得到启发,G2Tex算法输入是深度相机扫描的深度图集合和彩色图集合,借助基于RGB-D三维重建算法~\cite{LongYang2018SurfaceRV}利用稀疏序列融合(Sparse-Sequence Fusion,SSF)策略并加以改进,获取高可用、高质量的RGB-D序列重建TSDF模型,并用移动立方体算法抽取出网格模型。重建几何模型选帧策略依照如下公式进行:
\begin{align}
E(i)= \lambda_{1} E_{\mathrm{jit}}(i)+\lambda_{2} E_{\mathrm{dif}}(i)+\lambda_{3} E_{\mathrm{vel}}(i)+\lambda_{4}E_{\mathrm{cla}}(i)+E_{\mathrm{sel}}(i)
\end{align}
其中$E_{\mathrm{sel}}(i)$当前帧$i$是否作为三维重建的有效帧的控制项,$E_{\mathrm{jit}}(i)$衡量当前帧$i$与前一个有效帧的抖动强度比较,尽量保证选取平滑非抖动帧进行重建,$E_{\mathrm{dif}}(i$确保前后有效帧间有超过阈值重叠覆盖率,避免相机跟踪失败,$E_{\mathrm{vel}}(i)$度量相机运动速度,避免相机运动过快产生的模糊帧,$E_{\mathrm{cla}}(i)$是图像清晰度度量指标,衡量彩色图片清晰度以保证选择的彩色图像最清晰。经过重建算法后我们得到初始网格模型$M$。
\noindent\textbf{关键帧获取。}由于RGB-D相机是手持进行拍摄,拍摄过程中不可避免的出现抖动相机运动过快,往往会造成所拍摄图片发生模糊、失真等现象。为了根除这个现象,我们会根据 Crete 等人提出的图片的模糊度度量指标 ~\cite{FrederiqueCrete2007TheBE}评估每一帧。对于彩色图超过100张的数据集,经验上我们设置一个尺寸为5的滑动窗口,在每个窗口内选择模糊度最小的图片作为关键帧$KF$,根据数据集的大小灵活调整窗口尺寸。经过实验证明选择关键帧并不会影响优化效果而且会线性减少优化时间。 \par
\noindent\textbf{辅助视角选取。}受Huang等人~\cite{JingweiHuang2020AdversarialTO}的启发,我们为每个原始视角$T_s$选取一个辅助视角$T_t$,使得辅助视角重投影至原视角作为真例以此来监督渲染的图片。令$p_s$,$p_t$分别表示原始视角和目标视角的齐次坐标,则二者的对应关系可以描述为:
\begin{align}
	p_s\sim KT_{t\rightarrow s}D_tK^{-1}p_t
\end{align}
其中$D_t$表示目标视角的深度值。由于遮挡和深度图存在噪声的缘故,目标视口扭曲到原始视角因未对齐而产生残缺现象。为了防止残缺部分过大,我们为任意两个视口计算z方向夹角$\theta$,当$\theta\le15^{\circ}$时两个视角才被符合源视角-目标视角位姿对儿集合。

\subsection{相机参数优化}
在本节中,我们详细介绍相机优化原理与步骤。刚体变换的旋转表示有旋转矩阵、欧拉角、轴角表示、四元数等。我们使用旋转矩阵,一方面易于向其他表示转换,另一方面很容易在笛卡尔坐标系下变换顶点位置。令$ \left \{ \mathbf{I_i} \right \} $表示输入图像集合,$\left \{ \mathbf{T_i} \right \}$表示相机位姿集合,$\left \{ \mathbf{P_i} \right \}$表示图像对应的模型顶点集合。旋转矩阵定义为:
\begin{align}
	T = \left[\begin{array}{cc}
		R & t \\
		0 & 1
	\end{array}\right]
\end{align}
$T \in \mathrm{SE} (3)$,$R \in \mathrm{SO}(3)$并且$t\in\mathbb{R}^3$,其中$T \in \left \{ \mathbf{T_i} \right \}$为4$\times$4矩阵。\par 
三维重建模型定义在世界坐标系下,并以模型中心为坐标原点。经过刚体变换后,以相机所在位置为坐标原点,模型处于在相机正前方。然后我们只需要将相机坐标系下的模型顶点投影至二维图像平面上即可。根据相机成像模型定义刚性变换为$\mathbf{G}(p)=Tp, p \in \left \{ \mathbf{P_i} \right \}$,变换后的顶点齐次坐标为$\mathbf{g}=[g_x, g_y, g_z,g_w]^\top$。设投影图像平面变换为$\mathbf{U}$,我们按照以下公式计算图像平面中的像素位置:
\begin{align}
\mathbf{U}\left(g_{x}, g_{y}, g_{z}, g_{w}\right)=\left(\frac{g_{x} f_{x}}{g_{z}}+c_{x}, \frac{g_{y} f_{y}}{g_{z}}+c_{y}\right)^{\top}
\end{align}

其中,$f_x,f_y$表示相机的在$x,y$方向的焦距长度,$c_x,c_y$分别表示相机的光心坐标。这些参数可以利用相机标定获取。\par

在基于RGB-D的三维重建中一般使用迭代最近点(Iterative Closest Point,ICP)算法~\cite{besl1992method,chen1992object}及其衍生算法估计初始的相机位姿。由于存在噪声相机位姿估计存在累计误差,即使有回环检测也并不都能完全消除。使用不完美的相机位姿渲染图片,渲染图片和采集图片会发生错位现象。在传统的优化相机方案中,通常计算网格顶点重投影误差来优化相机投影矩阵$T$。设模型上一顶点$p$,初始颜色为$C(p)$(初始颜色,利用顶点投影后的像素颜色平均可得),定义$\Gamma(x,y) $为灰度图像上$(x,y)$位置的颜色值。顶点颜色与投影至视角$i$上的像素颜色值会存在差值,定义残差项为:
\begin{align}
	R_{i, \mathbf{p}}=C(\mathbf{p})-\Gamma_{i}\left(\mathbf{U}\left(\mathbf{G}\left(\mathbf{p}, \mathbf{T}_{i}\right)\right)\right)
\end{align}
同时定义优化目标为
\begin{align}
	E(\mathbf{T})=\sum_{i} \sum_{\mathbf{p} \in \mathbf{P}_{i}} R_{i, \mathbf{p}}^{2}
\end{align}
利用高斯牛顿优化方法~\cite{wedderburn1974quasi}对上述公式进行迭代求解,并且更新一次相机位姿$T$后,重新计算每个顶点的颜色$C(p)$。最终获得精确的相机位姿以及帧间一致的顶点颜色。由传统优化方法得到启发,我们借助于可微分渲染框架以投影顶点至图像平面,使用UV纹理保存顶点颜色,并使用$L_1$范数度量图像间的颜色差异,最后使用梯度下降算法朝向目标方向优化相机位姿矩阵$T$。\par
基于渲染能力和效率的考量,并且我们无需对模型表面材质进行额外建模。所以使用基于光栅化的渲染框架~\cite{ravi2020pytorch3d}和布林冯着色模型,渲染目标遵从以下目标:
\begin{align}
	C_i = (I_a + I_d) * P_i + I_s
\end{align}
其中$I_a,I_d,I_s$分别表示环境光、漫反射光和高光项,$C_i,P_i$分别表示彩色图像素和纹理像素。为了最大限度的还原具有保真度的纹理,我们只保留环境光,设置高光和漫反射光系数为0。对于相机外参T,我们采用以上介绍的投影和表示方法。 由于渲染框架中所有模块都是可微分的,并可以嵌入到深度网络中,从而实现端到端的训练。具体地,每一次渲染我们会随机选择某个视角,在该视角下生成彩色图$\tilde{I}_c$深度图$\tilde{I}_d$以及阴影图$\tilde{I}_s$即$I_c,I_d,I_s = Render(M_0|T_i)$。我们首先使用光度一致性损失来优化已选帧的相机位姿
\begin{align}
	E_c = \left \| I_c - \tilde{I}_c  \right \|_1 
\end{align}
然而仅仅具有光度一致性损失不足以约束相机位姿以朝着我们期望的方向优化,在某些场景中纹理较为单一或者纹理细节较少,相机仍会发生漂移现象。因此我们额外考虑深度图损失来增强几何一致性。
\begin{align}
	E_d = \left \| I_d - \tilde{I}_d  \right \|_1 
\end{align}
受SoftRas~\cite{ShichenLiu2019SoftRA}启发我们也采用轮廓损失来对几何进行约束。这个阴影损失函数为:
\begin{align}
	E_s = 1 - \frac{\left \| \hat{I_s}\otimes I_s  \right \|_1 }{\left \| \hat{I_s}\oplus  I_s- \hat{I_s}\otimes I_s \right \| }_1  
\end{align}
其中$\otimes $和$\oplus $分别是按元素计算的乘法运算符和运算符。最后我们按照每个损失对优化相机参数的贡献来赋予每个损失不同的权重。
\begin{align}
	L_T = \lambda_c E_c + \lambda_d E_d +\lambda_s E_s
\end{align}经验上我们设置$\lambda_c = 0.1$,$\lambda_d = 1$,$\lambda_s = 1$。\par
注意,在不同场景下我们可能微调优化相关的超参数,并且相机位姿本身没有真值可得。评估相机位姿是否准确的唯一方案就是利用优化后的相机位姿$T'$生成新的纹理图,并比较与原始纹理之间的清晰度,越清晰则优化效果越好。

\subsection{纹理重合成}
仅仅优化相机参数不足以保证网格上任意顶点投影至每个视角得到一致颜色,尤其是在网格重建误差较大情况下。不仅如此,在重建三维模型时一般会选用加权平均方法来抵抗重建过程中的噪声,虽然这种方法卓有成效,但是会造成过平滑的效果致使网格失去几何高频细节。我们同样也借助于可微分渲染方法恢复出几何表面的高频几何细节。我们为网格上每个顶点施加偏移量来矫正几何误差。\par

经过全局优化后,大部分的纹理块之间的纹理可以完全缝合。但是,由于重建模 型的一些区域存在明显的几何误差,所以经过全局优化并不能完全消除纹理之间的缝 隙。而且全局优化只是对纹理块之间的相机位姿 T 进行非刚性地校正,如果重建模 型足够精确所有的纹理块之间能够完美的对齐。不幸的是重建模型上无处不在的几何 误差和扭曲使得全局优化后纹理结果并没有达到全局一致的结果。所以我们需要对纹 理结果进行进一步的优化来完全消除纹理之间的缝隙,以得到高质量的纹理结果。在 局部优化中我们对纹理边界上顶点的纹理坐标进行调整使得纹理之间缝隙可以完全缝 合。

仅仅只矫正相机位姿是不够的,优化结果并不能完全消除噪声。单纯的优化相机

况且我们采用顶点加权融合方式生成纹理,仍存在模糊并且缺失细节。


由Huang等人~\cite{JingweiHuang2020AdversarialTO}得到启发我们用对抗生成网络学习重建模型和拍摄的彩色图片错位的容忍度。我们使用基于$\pi$GAN\cite{chanmonteiro2020pi-GAN}的网络用像素重生成管线合成纹理图。我们使用辅助视角重投影至原视角$I_c^{B\to A}$为真例监督可微分渲染生成的彩色图假例以更新纹理。相应地对抗损失函数定义如下:
\begin{align}
	L_{a d v}=\log D\left(\boldsymbol{I}_{A}, \boldsymbol{I}_{B \rightarrow A}\right)+\log \left(1-D\left(\boldsymbol{I}_{A}, \boldsymbol{I}_{A}^{D R, t}\right)\right) 
\end{align}
经过若干次迭代交替训练判别器$D$和纹理$P$后,判别器会识别渲染图片中存在的伪影模糊或者裂缝,纹理会重新生成像素以弥补渲染图片和真实图片之间错位现象,使得两者充分接近以愚弄判别器。经过优化后的纹理相比与用加权融合方法生成的纹理更加真实,更接近于真实拍摄的彩色图片。
\subsection{交替优化}
相似于之前工作,我们使用联合优化策略优化相机参数、几何模型和纹理。我们用相机-网格-纹理优化顺序进行。一方面是在相机和网格优化中纹理生成方法不同于神经网络的像素重生成方法,另一方面在矫正相机参数和网格后,对抗生成网络效果重生成的纹理更加贴近于真实世界场景。具体地我们采用外部循环方法分别优化参数集$(T,M,P)$。我们首先固定参数$(M,P)$,通过最小化损失函数$L_T$以优化每一帧的相机参数$T$至$T'$;其次,我们使用参数$(T',P)$最小化损失函数$L_M$优化几何模型$M$至$M'$;最后我们使用并固定前两次的结果参数$(T',M')$最小化对抗损失$L_{adv}$来优化纹理$P$至$P'$。\par
在时间和效率的权衡下我们重复外循环优化策略3次。并且我们遵从由粗到细的策略优化不同的目标。具体地,在每一次外部迭代$t\in \left \{ 1,2,3 \right \}$中我们用指数衰减方式控制内部迭代次数,即内部迭代次数为$\text{epoch}  =\frac{s}{2^{t-1}}$,其中s为每一个内部迭代初始的轮数。除此之外,学习率也会相应地进行减半操作。我们的经验发现,利用指数衰减测率,可以保证最终效果的同时显著减小优化时间。我们设置初始的优化相机参数、优化几何模型和纹理次数为50,50,100。

\section{实验}
\paragraph*{比较方法}
在这个部分我们会和最新和方法G2Tex~\cite{fu2018texture},JointTG~\cite{YanpingFu2020JointTA},ATO~\cite{JingweiHuang2020AdversarialTO},Intrinsic3d~\cite{RobertMaier2017Intrinsic3DH3}在公开的RGB-D数据集上并用他们在GitHub提供的源代码在进行比较。因为ADJOIN~\cite{9705143}没有公开代码可得所以不进行比较。然后在我们自己拍摄的RGB-D的数据集上进行比较,最后我们会展示消融研究以证明我们联合优化框架的有效性。我们整个实验结果都在单个NVIDIA GeForce RTX3090 24GB服务器上计算得出。


\paragraph*{评估指标}
为了对所生成的纹理进行定量的研究,我们采用多种指标来衡量纹理和真实图片之间的差异。因为没有准确的指标来直接衡量生成的纹理图集,所以我们仍旧将纹理渲染为图像,然后利用经典的图像的评估指标,如峰值信噪比(PNSR)、结构相似性(SSIM)和感知相似度(LPIPS)来侧面衡量纹理清晰度和保真度。注意,我们使用所有视角评估的平均值作为最后的结果。
\subsection{在公共数据集上进行评估}
我们首先在ATO~\cite{JingweiHuang2020AdversarialTO}发布的椅子数据集上数据集上数据集上进行实验比较。如图X所示,可以看出我们的方法得到了最佳的效果,ATO采用对抗神经网络方法合成纹理图,虽然能够容忍较小的相机误差和几何误差,但是在局部地方仍然有模糊现象。G2Tex采用面投影方法,并优化了相机位姿,能够产生的纹理整体比较清晰,但是局部地方会有明显裂缝出现,这一点纹理比较丰富的数据集上更加明显。我们额外进行了定量评估,生成模型后我们在所有视角下评估定量指标最后取平均值展示。如表所示,我们的结果在椅子数据集上的定量指标(如上所示的)均超过了,最新的纹理优化方法。\par
其次我们在Intrinsic3D~\cite{RobertMaier2017Intrinsic3DH3}发布的数据集上进行评估,我们和intrinsic3D方法和JointTG方法,进行实验对比。这两个方法都分别实现了一个联合优化框架,同时优化纹理和几何模型。如图XX所示,我们的方法在纹理优化和几何优化上都取得了更佳的效果。由于Intrinsic3D基于阴影恢复形状(SFS),恢复几何细节方面效果非常好,但是由于SFS固有的缺陷,intrinsic3d很容易出现纹理复制伪影,尤其是当视角非常稀疏的情况。JoinTG的看起来视觉效果非常好,但是在几何和纹理细节方面,我们的结果比JoinTG更好,尤其是在人物手部眼睛方面,我们的方法恢复的细节更多,看起来更接近于拍摄图片。\par
最后,我们比较各种基线方法在zhou等人~\cite{Zhou2018}发布的Fountain数据集,该数据集视角非常稀疏。在此数据集上,我们的方法可以很好的处理标准化数据集,如图所示。我们的方法在几何细节和纹理细节方面对比其他方法展示出优越性。如图红色框中所示,我们的方法能恢复出清晰的喷泉模型文字,产生一致的纹理。其他方法在局部区域要么会产生模糊伪影要么会产生不一致的边界如G2Tex。我们额外实现了定量评估,如表所示,我们的方法对重建误差进行矫正,再用像素重生成管线合成纹理图,对几何误差和相机位姿误差更具有鲁棒性。经过联合优化后重建出的纹理,渲染的图片更接近于真值。\par
\subsection{在我们自己拍摄的数据上进行评估}
\subsection{消融研究}
在这个部分,我们将研究方法中各个部分的有效性。为了更有说服力,我们选择真实场景中的数据集Fountain。首先我们移除各个部分,没有优化相机位姿,没有优化几何模型,没有对抗生成网络,不采用自适应细分的情况,最后我们还讨论自适应细分方法,质心细分和基于边的细分效果。\par
如图所示,我们展示了移除的各个部分效果。没有矫正相机位姿情况下,渲染图片和真实图片会有错位现象,借助于对抗生成我们仍都能得到清晰的纹理,但是缺乏细节。没有几何细化,看起来对纹理生成效果最小,但是会在细节处产生轻微的伪影。没有对抗生成网络时,无论是纹理细节还是整体视觉效果都明显下降,并在定量评估中我们同样得出了一致的结论。\par
我们采用自适应细分方法,在纹理丰富处产生更多的顶点,保证视觉效果的同时减少数据量。如图所示,我们分别基于面和基于边的做法,虽然基于边的细分能产生更加均匀的效果,但是会导致未细分面片和已细分面片的公共边上的顶点产生读书不平衡,而基于面的做法保持规整。当优化几何模型时,基于边的细分容易导致拓扑结构的破坏,即三角形变形为四边形从而产生奇怪的几何模型。
\noindent \textbf{运行时间}
我们的优化时间随着数据集帧数线性增长,我们的方法在ATO发布的数据集上每个场景花费平均63分钟,其中优化相机位姿、几何、纹理时间占比分别为30,40,30。


\chapter{基于RGB-D三维重建几何模型与纹理优化}
\section{本章内容简介}

由于三维重建过程中存在着大量的噪声以及量化误差和估计误差等因素导致三维重建模型无法如真实场景般精确,正是因为这些误差存在使得纹理图像无法与模型对齐,造成不可避免的模糊伪影或者裂缝现象。直观地,要想得到高保真纹理模型必须消除或者矫正这些误差项,但是现有的方法只将注意力放在纹理图像上,生成纹理图像以适配几何模型。忽略了几何与纹理是一个整体,单纯的优化其中一项对于纹理提升有限并使得另一项成为制约获取理想纹理模型的瓶颈。本文在基于可微分渲染纹理优化(第三章所述方法)基础上提出结合自适应细分的联合优化框架,并对纹理、相机位姿、几何模型分别进行优化,然后我们又提出联合优化方案使得更快的达成优化目标,避免共同优化多个变量陷入局部最小值无法收敛,最终我们的方案能够生成精细的几何模型,准确的相机位姿,清晰的纹理。相比于传统的联合方案,我们的方法具有可扩展性和鲁棒性,能够胜任各种场景环境,如存在灯光、弱纹理条件等。实验证明我们在灯光干扰,公共数据集和我们收集的数据集上均取得良好的表现证明了我们方法的有效性。

\section{引言}


得益于消费机RGB-D相机,如Microsoft Kinect、Intel RealSense或Google Tango等商用RGB-D传感器被广泛使用,3D场景的重建得到了极大的关注人们能够很容易的对真实世界的场景进行建模得到场景和物体的三维模型,基于RGB-D相机的静态场景~\cite{RichardNewcombe2011KinectFusionRD,ThomasWhelan2012KintinuousSE,ThomasWhelan2015ElasticFusionDS,SungjoonChoi2015RobustRO,VictorAdrianPrisacariu2017InfiniTAMVA,AngelaDai2016BundleFusionRG}和动态场景重建~\cite{newcombe2015dynamicfusion,innmann2016volumedeform,yu2017bodyfusion,yu2018doublefusion,xu2019unstructuredfusion,su2020robustfusion,slavcheva2017killingfusion,gao2019surfelwarp,dou2016fusion4d}技术也发生了前所未有的进步。即使如此,三维重建结果无法满足虚拟现实、混合现实、动画、影视、游戏等领域的需求,一方面是重建几何模型存在瑕疵;另一方面,即使几何模型结果满足基本需求模型外观也无法达到与照片相媲美的程度,很难使人满意也无法直接应用。\par

一个高保真的纹理映射模型应该符合两个标准:精确的重建模型、清晰且全局一致的纹理。然而这两个标准常常被以下因素干扰。首先,深度相机采集的深度并非绝对准确,当测量距离大于一定阈值时会存在大量误差,而且在室内也会受光源影响使得度量误差增大。然后,对于几何模型现有的三维重建算法为了抵抗噪声在重建表面过程中用加权平均方法,但是会抹去几何模型的高频细节使得重建模型整体较为完美但是细节处存在瑕疵。再次,由于各种误差存在,即使存在回环检测、位姿图优化等算法,相机位姿也只能保证相对准确。以上各种因素累加后采集的图像和几何模型就无法准确对齐,要想获取高质量、高保真的纹理需要进一步做优化。\par


先前的工作采用各种方法试图解决以上所述问题,一般通过合成纹理图像~\cite{bi2017patch}、调整相机位姿~\cite{zhou2014color}或者细分网格顶点~\cite{ChengleiWu2014RealtimeSR}来生成更高质量的三维模型。有的工作不止着眼于单个组件的优化,通常在先前工作基础上额外提出优化策略使得纹理映射结果更加逼真。如G2Tex~\cite{fu2018texture}采用在基于面投影方法上额外使用从全局到局部的相机位姿矫正方案成功消除了纹理边界处的大量细缝,可以生成照片级纹理图像取得了良好的效果,seamless~\cite{fu2021seamless}也采用面投影方式,并额外采用块合成的方法在纹理边界处重新合成图案以消除纹理。大部分工作聚焦于如何改进或者生成纹理图像以弥补重建模型的误差,但是补偿能力有限只适用于重建模型误差较小情况,而且未考虑纹理、相机位姿、几何模型之间的耦合性关系,在优化场景选择上有限制。Intrinsic3D~\cite{RobertMaier2017Intrinsic3DH3}是首个基于阴影恢复形状(Shape From Shading,SFS)方法,对场景中的几何模型、纹理、相机位姿以及场景光照进行联合优化,但是由于SFS方案会产生纹理复制现象制约了Intrinsic3D方法的应用。最近fu等人提出JointTG~\cite{YanpingFu2020JointTA}方案对几何、相机位姿、纹理分别进行迭代优化,再采用分层方案代替混合优化策略成功解决Intrinsic3D纹理复制问题并且速度更快。但其复杂的框架分别针对几何、纹理和相机姿势使用不同的优化方案和目标函数,使得其可扩展性和健壮性较差,特别是在重建误差较大的情况下。\par


受可微分渲染在三维单视图模型重建~\cite{liu2020general}~\cite{ShichenLiu2019SoftRA}中的成功启发,我们将对抗生成网络与可微分渲染结合,提出一个基于可微分渲染的联合优化深度学习框架,同时对相机位姿与纹理,细化和重建模型进行联合。克服现有只对一个因素进行优化而难以达到全局最优解的障碍,并最终得到带有高质量几何细节和高保真纹理的三维重建结果。我们联合框架核心是可微分渲染模块,从纹理模型渲染至图像需要几何模型、纹理、相机位姿的共同参与,很容易使得梯度在各个组件之间进行更新传递,而且使用共同的图像间的损失来优化不同的组件,大大提高了可扩展性和灵活性。不仅如此,在优化几何模型的同时,为了突出恢复模型的高频细节,我们额外引入自适应质心细分方法,即只细分纹理丰富区域的几何模型。相比于平凡地优化几何模型,我们的方法在能恢复出几何模型的细节方面具有优越性。为了加快收敛速度与提高优化效果,我们提出了交替优化方案,分层迭代更新每一个组件,降低模块之间的耦合性,提高优化过程中的稳定性。\par

我们的实验表明,与最先进的方法相比,我们的联合优化框架相比于最新纹理优化方法在合成数据上的定量和真实数据上的定性上都产生了显著的性能提高。也展示了我们在恢复出三维重建模型高质量的几何细节和高保真的纹理细节方面具有优势。\par
总的来说,本章主要包括以下几个贡献:\par
\begin{enumerate}
\item 我们提出了基于可微分渲染的新的联合优化方案,将几何模型、纹理图像和相机位姿共同纳入统一的框架内,对每个组件进行交替优化,最终得到高质量的几何模型与高保真纹理的图像。
\item 我们采用对抗生成网络进行纹理图像重合成以增加真实感,渲染模块可以灵活嵌入神经网络中,并且网络结构可以灵活选择、优化过程灵活选择,提升了可扩展性。
\item 在优化网格顶点时,我们提出自适应细分方法,只在纹理丰富区域进行细分,减少代价同时达到与整体细分一致的效果。
\end{enumerate}


\section{相关工作}
\subsection{纹理优化}
为了恢复三维重建模型高质量的几何细节或高保真的纹理细节,近十几年,在三维重建领域做出了许多努力。我们从影响纹理优优化结果的不同因素:不精确的相机位姿、几何模型和纹理出发,按不同的优化思路把相关的算法分为一下几类。\par
\vspace*{2mm}\noindent{\bf 基于投影的方法:}它为每个面片选择最佳视角Lempitsky等人~\cite{lempitsky2007seamless}使用成对儿的马尔可夫随机场为每个三角面选择最优图像。这种方法面临一个具有挑战性的问题,即如何减轻相邻纹理之间的视觉接缝。Waechter等人~\cite{waechter2014let}提出了一种全局色彩调整算法,以减少由于视图投影造成的视觉破裂。Fu等人~\cite{fu2018texture} 提出了一种全局到局部的非刚性优化方法来调整摄像机的姿态漂移,并纠正几何误差引起的纹理坐标漂移。\par
\vspace*{2mm}\noindent{\bf 基于wrap的方法:}基于wrap的方法能够抵抗几何误差和相机漂移引起的对准误差问题。Zhou等人~\cite{zhou2014color}设计了一个纹理映射框架,通过局部图像扭曲来纠正相机的姿态和几何误差。但该方法需要对网格模型进行细分,这将大大增加数据量,限制其应用范围。\par
基于块合成的方法:Bi等人~\cite{bi2017patch}采用基于 patch 的图像合成方法来生成一个新的完全对齐的目标纹理图像 来消除纹理图像之间的不对齐,从而避免纹理结果的模糊和重影。\par
\vspace*{2mm}\noindent{\bf 基于联合优化的方法:}Robert Maier等人~\cite{RobertMaier2017Intrinsic3DH3}提出了一种基于SFS(shape-from-shading)和空间变化的球谐光照函数的子体优化方法,同时优化几何、纹理、相机姿态和场景照明。可以获得纹理一致的高质量三维重建。但该方法依赖于SFS,需要分解场景的光照,容易导致纹理拷贝问题。最近工作中Fu等人~\cite{YanpingFu2020JointTA}提出根据颜色和几何一致性以及高频法线线索对重建网格进行优化,有效地克服了SFS产生的纹理拷贝问题,从而得到了更加高质量的重建结果。\par
\vspace*{2mm}\noindent{\bf 基于深度学习的方法:}Huang等人~\cite{JingweiHuang2020AdversarialTO}使用从弱监督视图中获得的条件对抗损失为近似表面生成逼真的纹理,使用基于学习的方法训练纹理目标函数,以保持对摄像机姿态和几何畸变的鲁棒性。与我们相似的工作zhang等人~\cite{9705143}借助于可微分渲染方法提出了一种联合优化方法,将几何、纹理和相机姿态共同纳入一个统一的优化框架中,并采用一种自适应交织策略,提高优化的稳定性和效率。与我们的方法类似但是我们的方法在优化几何时采用自适应性细分,从而当几何误差过大时,我们的方法仍能恢复出到几何模型的细节,并能获取高保真的纹理。
\subsection{可微分渲染}
可微分渲染是通过梯度设计,能够将渲染输出的梯度方向传播至三维实体,从而弥补了二维和三维之间的差距,同时可以允许神经网路在操纵渲染图像的同时优化三维实体,无需额外的三维标注。基于不同的三维实体如体素、点云、SDF、网格有不同的可微分渲然方法,因为网格能够表达三维模型的拓扑结构,又无需关注三维实体内部的构成,这种表示方式不仅灵活而且节省内存,所以本文只关注基于网格的可微分渲染。
Loper和Black~\cite{MatthewLoper2014OpenDRAA}设计了基于网格的通用框架OpenDR,近似可微渲染器。Kato等人~\cite{MatthiasNiener2013Realtime3R}提出了一种神经3D网格渲染器,用手工设计的函数来近似光栅化后向梯度。SoftRas~\cite{ShichenLiu2019SoftRA}以概率方式将每个像素分配给网格的所有面,保证前向后向操作均可微。pytorch3d~\cite{ravi2020pytorch3d}基于SoftRas方法,设计出通用的基于网格的可微分框架,不仅方便地更新几何模型而且还能操作纹理贴图。我们的方法基于pytorch3d,借助于可微分渲染框架,将梯度反向传播并更新场景中相机位姿,几何和纹理贴图。注意我们并未考虑场景中其他参数,例如材质,灯光等。\par
更多关于可微分渲染的详细介绍请参阅综述\cite{HiroharuKato2020DifferentiableRA}。

\section{算法流程}

我们的方法旨在通过RGB-D摄像机获得具有精细几何细节三维模型和高保真的纹理贴图,为此我们提出了一个联合优化算法。优化几何模型,恢复出几何模型高频的几何细节;优化纹理,使之重生成高保真度的纹理;优化相机位姿,矫正估计不准的相机位姿,图X展示了我们方法流程。\par
这个部分,我们会阐述我们所提出方法的详细步骤。令$P$表示纹理、$M$表示几何模型(网格表示)、$T$表示相机外参、$K$表示相机内参。$I_c$,$I_D$,$I_S$分别表示手持RGB-D拍摄的彩色图、纹理图以及利用标准的计算机图形学管线生成的阴影图。纹理、几何模型和相机位姿和我们的最终目标(即恢复出清晰具有保真度的纹理)高度相关,而且三者之间具有耦合性。值得注意的是渲染本身就需要纹理、相机和几何模型的共同参与,可以将其纳入一个统一的优化框架,这与我们的优化目标高度契合。我们借助于pytorch3d ~\cite{ravi2020pytorch3d}可微分渲染框架,依据pytorch3d渲染流程我们可以制定更加合理的优化策略即每个部分分别进行交替迭代优化。下一个部分我会详细描述在统一的优化框架内纹理、相机位姿和网格的各自优化过程。\par
\subsection{数据处理}
\noindent\textbf{输入。}RGB-D数据我们使用消费级深度相机在固定曝光和白平衡模式下拍摄。我们利用经典的VoxelHashing框架~\cite{MatthiasNiener2013Realtime3R}获取初始的网格模型。对于初始纹理,我们利用加权平均策略将顶点从多个视口反投影至纹理图集上以生成初始纹理。\par
\noindent\textbf{关键帧获取。}由于RGB-D相机是手持进行拍摄,拍摄过程中不可避免的出现抖动相机运动过快,往往会造成所拍摄图片发生模糊、失真等现象。为了根除这个现象,我们会根据 Crete 等人提出的图片的模糊度度量指标 ~\cite{FrederiqueCrete2007TheBE}评估每一帧。我们设置一个尺寸为5的滑动窗口,在每个窗口内选择模糊度最小的图片作为关键帧$KF$。 \par
\noindent\textbf{辅助视角选取。}收Huang等人~\cite{JingweiHuang2020AdversarialTO}的启发,我们为每个原始视角$T_s$选取一个辅助视角$T_t$,使得辅助视角重投影至原视角作为真例以此来监督渲染的图片。令$p_s$,$p_t$分别表示原始视角和目标视角的齐次坐标,则二者的对应关系可以描述为:
\begin{align}
	p_s\sim KT_{t\rightarrow s}D_tK^{-1}p_t
\end{align}
其中$D_t$表示目标视角的深度值。由于遮挡和深度图存在噪声的缘故,目标视口扭曲到原始角会发生未对齐和残差现象。为了防止残缺部分过大,我们为任意两个视口计算z方向夹角$\theta$,当$\theta\le15^{\circ}$时两个视角才被加入到位姿对儿集合中。

\subsection{相机参数优化}
在基于RGB-D的三维重建中一般使用光束平差法估计初始的相机位姿。由于存在噪声相机位姿估计存在累计误差,即使有回环检测也并不都能完全消除。使用不完美的相机位姿渲染图片,渲染图片和采集图片会发生错位现象。借助于可微分渲染框架我们可以朝着期望的方向对相机位姿进行优化。\par
渲染模型选择上,我们使用标准的图像渲染流程即光栅化和布林冯着色模型。渲染目标遵从以下目标:
\begin{align}
	C_i = (I_a + I_d) * P_i + I_s
\end{align}
其中$I_a,I_d,I_s$分别表示环境光、漫反射光和高光项,$C_i,P_i$分别表示彩色图像素和纹理像素。为了最大限度的还原具有保真度的纹理,我们只保留环境光。对于相机外参T,令$T = (R,t)\in \mathrm{SE} (3),R_i \in \mathrm{SO}(3)$并且$t\in\mathbb{R}^3$。 
由于我们的框架中所有模块都是可微分的,可以端到端的进行训练。每一次渲染我们会随机选择某个视角,在该视角下生成彩色图$\tilde{I}_c$深度图$\tilde{I}_d$以及阴影图$\tilde{I}_s$即$I_c,I_d,I_s = Render(M_0|T_i)$。我们首先使用光度一致性损失来优化关键帧的相机位姿
\begin{align}
	E_c = \left \| I_c - \tilde{I}_c  \right \|_1 
\end{align}
然而仅仅具有光度一致性损失不足以约束相机位姿以朝着我们所想的方向优化,在某些场景纹理单一或者纹理较少,相机仍会发生漂移现象。因此我们额外考虑几何损失来增强几何一致性。
\begin{align}
	E_d = \left \| I_d - \tilde{I}_d  \right \|_1 
\end{align}
受SoftRas~\cite{ShichenLiu2019SoftRA}启发我们也采用轮廓损失来对几何进行约束。这个阴影损失函数为:
\begin{align}
	E_s = 1 - \frac{\left \| \hat{I_s}\otimes I_s  \right \|_1 }{\left \| \hat{I_s}\oplus  I_s- \hat{I_s}\otimes I_s \right \| }_1  
\end{align}
其中$\otimes $和$\oplus $分别是按元素计算的乘法运算符和运算符。最后我们按照每个损失对优化相机参数的贡献来赋予每个损失不同的权重。
\begin{align}
	L_T = \lambda_c E_c + \lambda_d E_d +\lambda_s E_s
\end{align}经验上我们设置$\lambda_c = 0.1$,$\lambda_d = 1$,$\lambda_s = 1$。

摄影机校正摄影机姿势Ti由Euler角度表示中的旋转和平移向量组成,每帧产生6个参数,这些参数使用BundleFusion姿势进行初始化,并在关节优化期间进行细化。受[63]的启发,添加了6层RELU MLP形式的相机像素空间的附加2D变形场,以考虑输入图像中可能的扭曲或相机固有参数的不准确。请注意,此校正域对于每个帧都是相同的。在优化过程中,相机光线首先使用在使用摄影机姿势Ti变换到世界空间之前,从变形场检索的2D向量。最后,在校正后的射线上绘制采样点。

\subsection{结合自适应细分的网格优化}
仅仅优化相机参数不足以保证网格上任意顶点投影至每个视角得到一致颜色,尤其是在网格重建误差较大情况下。不仅如此,在重建三维模型时一般会选用加权平均方法来抵抗重建过程中的噪声,虽然这种方法卓有成效,但是会造成过平滑的效果致使网格失去几何高频细节。我们同样也借助于可微分渲染方法恢复出几何表面的高频几何细节。我们为网格上每个顶点施加偏移量来矫正几何误差。\par
仅仅只更新顶点位置不足以保证几何模型细节突出,因为几何模型本身就过于平滑。我们采用质心细分网格方法增加三角形面片数目,一方面能使得模型更加契合真是世界场景,另一方面能减小顶点优化时发生漂移现象。\par
即使如此,细分网格代价是巨大的,增加顶点面片数目会增加渲染时间,并且在一些含有平面较多且纹理单一的几何模型上细分视觉效果并不明显。我们经验地发现优化几何模型时,顶点移动频繁发生在纹理丰富的地方,而纹理较为单一时顶点移动并不明显。因此我们建议根据场景本身的纹理丰富程度来决定是否要细分的程度。具体的,我们遵循一下步骤:

\begin{enumerate}
	\item 用soble算子提取所有视角的梯度图$\nabla g_i$,作为细分面片的依据。
	\item 计算几何模型上每个面片$f_j$投影至每个可见视口$i$的面积$A_{ij}$,并求出面积总和$\sum_{i}^{n} A_{ij}$。
	\item 计算每个面片$f_j$采样概率$\sum_{i}^{n} A_{j} / \sum_{j}^{m}\sum_{i}^{n} A_{ij}$,其中m为几何模型中面片数量,n为视角数量。
	\item 按照概率对所有面片进行无放回随机抽样,并设定采样概率阈值,在阈值概率之上面片才会被选取。
	\item 对所选择面片$f_j$进行质心细分。
\end{enumerate}

我们在第四步随机抽样时,为了防止太多的面片不在查询集合中,我们使用中位数而不是平均值作为采样阈值。细分完成后,我们仍用同样的方式增加对应的纹理坐标,以保持顶点和纹理坐标的对应关系。\par
为了防止优化过程出现异常情况,如裂缝现象。我们先用meshlab~\cite{LocalChapterEvents:ItalChap:ItalianChapConf2008:129-136}剔除重复顶点和面片以及零面积面片,以保证几何模型的规范性。优化几何模型时仅仅有图像间损失是远远不够的,因为每次迭代中顶点移动不受约束,会破坏网格模型的拓扑结构,导致退化的三角形。因此我们在图像损失基础上增加了几何正则化项,拉普拉斯项、法线一致性项和$L2$项。\par
拉普拉斯项定义为顶点坐标减去其临近顶点的加权和,在优化过程中可以保持局部几何特征不变。在优化过程中,拉普拉斯损失项可以帮助模型收敛到一个更加平滑的解,从而提高几何模型优化的准确度和稳定性。它也可以有效抑制因噪声或不规则采样而产生的噪点,使得优化结果更加符合实际。令$V \in \mathbb{R}^{n\times3}$为存储顶点位置的矩阵,则拉普拉斯项定义为:
\begin{align}
	E_l = \sum_{i=1}^{n}\left \| LV \right \|_2 
\end{align}
其中$L\in\mathbb{R}^{n\times n} $是网格的拉普拉斯图。\par
在优化过程中为了防止顶点偏移太远我们用$L_2$正则化项来约束顶点偏移量。其中$\mathbf{v}_{i}$为初始网格的顶点坐标,$\widetilde{\mathbf{v}}_{i}$为当前网格顶点的坐标。
\begin{align}
	E_{r}=\sum_{i}^{n}\left\|\mathbf{v}_{i}-\widetilde{\mathbf{v}}_{i}\right\|^{2}
\end{align}
最后,为了保证网格具有平滑性,我们额外使用了网格法线一致性损失。即利用余弦相似度计算相邻面片的法线一致性。
\begin{align}
	E_n= \frac{1}{|\overline{\mathcal{F}}|} \sum_{(i, j) \in \overline{\mathcal{F}}}\left(1-\mathbf{n}_{i} \cdot \mathbf{n}_{j}\right)^{2}
\end{align}
其中$\overline{\mathcal{F}}$代表共享一条边的相邻三角形面片的集合。$i$,$j$表示任意一对儿三角形面片例如$f_i$,$f_j$的法线。\par
最终网格重建损失项表示如下:
\begin{align}
	L_M = L_T + \lambda_l E_l +\lambda_r E_r+\lambda_n E_n
\end{align}
在实际优化中我们分别设置$\lambda_l = 1000$,$\lambda_r = 1000$,$\lambda_n = 10$。
\subsection{纹理重合成}
仅仅只矫正相机位姿和网格顶点位置是不够的,优化结果并不能完全消除噪声。况且我们采用顶点加权融合方式生成纹理,仍存在模糊并且缺失细节。由Huang等人~\cite{JingweiHuang2020AdversarialTO}得到启发我们用对抗生成网络学习重建模型和拍摄的彩色图片错位的容忍度。我们使用基于$\pi$GAN\cite{chanmonteiro2020pi-GAN}的网络用像素重生成管线合成纹理图。我们使用辅助视角重投影至原视角$I_c^{B\to A}$为真例监督可微分渲染生成的彩色图假例以更新纹理。相应地对抗损失函数定义如下:
\begin{align}
	L_{a d v}=\log D\left(\boldsymbol{I}_{A}, \boldsymbol{I}_{B \rightarrow A}\right)+\log \left(1-D\left(\boldsymbol{I}_{A}, \boldsymbol{I}_{A}^{D R, t}\right)\right) 
\end{align}
由于生成模型的发展,图像重建和合成已经取得了显著的进展。尽管如此,真实图像和生成图像之间仍可能存在差距,特别是在频域。在这项研究中,我们表明,缩小频域的差距可以进一步改善图像重建和合成质量。我们提出了一种新的焦点频率损失,它允许模型通过对容易合成的频率分量进行降权来自适应地聚焦于难以合成的频率分量。该目标函数是对已有空间损失的补充,对由于神经网络固有的偏差造成的重要频率信息的损失提供了巨大的阻抗。
\begin{align}
	L_\mathrm{FFL}=\frac{1}{M N} \sum_{u=0}^{M-1} \sum_{v=0}^{N-1} w(u, v)\left|F_{r}(u, v)-F_{f}(u, v)\right|^{2}
\end{align}

经过若干次迭代交替训练判别器$D$和纹理$P$后,判别器会识别渲染图片中存在的伪影模糊或者裂缝,纹理会重新生成像素以弥补渲染图片和真实图片之间错位现象,使得两者充分接近以愚弄判别器。经过优化后的纹理相比与用加权融合方法生成的纹理更加真实,更接近于真实拍摄的彩色图片。
\subsection{交替优化}
相似于之前工作,我们使用联合优化策略优化相机参数、几何模型和纹理。我们用相机-网格-纹理优化顺序进行。一方面是在相机和网格优化中纹理生成方法不同于神经网络的像素重生成方法,另一方面在矫正相机参数和网格后,对抗生成网络效果重生成的纹理更加贴近于真实世界场景。具体地我们采用外部循环方法分别优化参数集$(T,M,P)$。我们首先固定参数$(M,P)$,通过最小化损失函数$L_T$以优化每一帧的相机参数$T$至$T'$;其次,我们使用参数$(T',P)$最小化损失函数$L_M$优化几何模型$M$至$M'$;最后我们使用并固定前两次的结果参数$(T',M')$最小化对抗损失$L_{adv}$来优化纹理$P$至$P'$。\par
在时间和效率的权衡下我们重复外循环优化策略3次。并且我们遵从由粗到细的策略优化不同的目标。具体地,在每一次外部迭代$t\in \left \{ 1,2,3 \right \}$中我们用指数衰减方式控制内部迭代次数,即内部迭代次数为$\text{epoch}  =\frac{s}{2^{t-1}}$,其中s为每一个内部迭代初始的轮数。除此之外,学习率也会相应地进行减半操作。我们的经验发现,利用指数衰减测率,可以保证最终效果的同时显著减小优化时间。我们设置初始的优化相机参数、优化几何模型和纹理次数为50,50,100。

\section{实验}
\paragraph*{比较方法}
在这个部分我们会和最新和方法G2Tex~\cite{fu2018texture},JointTG~\cite{YanpingFu2020JointTA},ATO~\cite{JingweiHuang2020AdversarialTO},Intrinsic3d~\cite{RobertMaier2017Intrinsic3DH3}在公开的RGB-D数据集上并用他们在GitHub提供的源代码在进行比较。因为ADJOIN~\cite{9705143}没有公开代码可得所以不进行比较。然后在我们自己拍摄的RGB-D的数据集上进行比较,最后我们会展示消融研究以证明我们联合优化框架的有效性。我们整个实验结果都在单个NVIDIA GeForce RTX3090 24GB服务器上计算得出。


\paragraph*{评估指标}
为了对所生成的纹理进行定量的研究,我们采用多种指标来衡量纹理和真实图片之间的差异。因为没有准确的指标来直接衡量生成的纹理图集,所以我们仍旧将纹理渲染为图像,然后利用经典的图像的评估指标,如峰值信噪比(PNSR)、结构相似性(SSIM)和感知相似度(LPIPS)来侧面衡量纹理清晰度和保真度。注意,我们使用所有视角评估的平均值作为最后的结果。
\subsection{在公共数据集上进行评估}
我们首先在ATO~\cite{JingweiHuang2020AdversarialTO}发布的椅子数据集上数据集上数据集上进行实验比较。如图X所示,可以看出我们的方法得到了最佳的效果,ATO采用对抗神经网络方法合成纹理图,虽然能够容忍较小的相机误差和几何误差,但是在局部地方仍然有模糊现象。G2Tex采用面投影方法,并优化了相机位姿,能够产生的纹理整体比较清晰,但是局部地方会有明显裂缝出现,这一点纹理比较丰富的数据集上更加明显。我们额外进行了定量评估,生成模型后我们在所有视角下评估定量指标最后取平均值展示。如表所示,我们的结果在椅子数据集上的定量指标(如上所示的)均超过了,最新的纹理优化方法。\par
其次我们在Intrinsic3D~\cite{RobertMaier2017Intrinsic3DH3}发布的数据集上进行评估,我们和intrinsic3D方法和JointTG方法,进行实验对比。这两个方法都分别实现了一个联合优化框架,同时优化纹理和几何模型。如图XX所示,我们的方法在纹理优化和几何优化上都取得了更佳的效果。由于Intrinsic3D基于阴影恢复形状(SFS),恢复几何细节方面效果非常好,但是由于SFS固有的缺陷,intrinsic3d很容易出现纹理复制伪影,尤其是当视角非常稀疏的情况。JoinTG的看起来视觉效果非常好,但是在几何和纹理细节方面,我们的结果比JoinTG更好,尤其是在人物手部眼睛方面,我们的方法恢复的细节更多,看起来更接近于拍摄图片。\par
最后,我们比较各种基线方法在zhou等人~\cite{Zhou2018}发布的Fountain数据集,该数据集视角非常稀疏。在此数据集上,我们的方法可以很好的处理标准化数据集,如图所示。我们的方法在几何细节和纹理细节方面对比其他方法展示出优越性。如图红色框中所示,我们的方法能恢复出清晰的喷泉模型文字,产生一致的纹理。其他方法在局部区域要么会产生模糊伪影要么会产生不一致的边界如G2Tex。我们额外实现了定量评估,如表所示,我们的方法对重建误差进行矫正,再用像素重生成管线合成纹理图,对几何误差和相机位姿误差更具有鲁棒性。经过联合优化后重建出的纹理,渲染的图片更接近于真值。\par
\subsection{在我们自己拍摄的数据上进行评估}
\subsection{消融研究}
在这个部分,我们将研究方法中各个部分的有效性。为了更有说服力,我们选择真实场景中的数据集Fountain。首先我们移除各个部分,没有优化相机位姿,没有优化几何模型,没有对抗生成网络,不采用自适应细分的情况,最后我们还讨论自适应细分方法,质心细分和基于边的细分效果。\par
如图所示,我们展示了移除的各个部分效果。没有矫正相机位姿情况下,渲染图片和真实图片会有错位现象,借助于对抗生成我们仍都能得到清晰的纹理,但是缺乏细节。没有几何细化,看起来对纹理生成效果最小,但是会在细节处产生轻微的伪影。没有对抗生成网络时,无论是纹理细节还是整体视觉效果都明显下降,并在定量评估中我们同样得出了一致的结论。\par
我们采用自适应细分方法,在纹理丰富处产生更多的顶点,保证视觉效果的同时减少数据量。如图所示,我们分别基于面和基于边的做法,虽然基于边的细分能产生更加均匀的效果,但是会导致未细分面片和已细分面片的公共边上的顶点产生读书不平衡,而基于面的做法保持规整。当优化几何模型时,基于边的细分容易导致拓扑结构的破坏,即三角形变形为四边形从而产生奇怪的几何模型。
\noindent \textbf{运行时间}
我们的优化时间随着数据集帧数线性增长,我们的方法在ATO发布的数据集上每个场景花费平均63分钟,其中优化相机位姿、几何、纹理时间占比分别为30,40,30。


\renewcommand{\chaptermark}[1]{\markboth{#1}{}}
% 总结与展望
% !TeX root = ../Template.tex
% 总结
\summary
\chapter{总结与展望}

%==============================
\section{工作总结}
本文首先介绍了基于RGB-D相机三维重建与纹理优化的背景与意义,以及纹理映射技术目前存在的技术挑战。然后,介绍了三维重建与纹理映射的相关工作做了总结与回顾。接着对于纹理优化相关技术做了梳理与总结。并且针对目前纹理优化中纹理存在模糊伪影以及裂缝现象我们提出基于学习的纹理优化方案,即利用可微分渲染技术矫正相机位姿,借助像素重生成管线重新生成纹理图片。为了保证在几何模型重建误差较大的情况下仍能获得高保真的纹理,我们提出一个联合优化算法对RGB-D相机重建的相机位姿、几何模型、纹理细节进行联合优化,最终获得高质量的几何模型和高保真的纹理。综上所述,本文贡献点有两个。\par
提出基于可微分渲染的相机位姿优化与纹理合成算法,获得清晰纹理。首先利用渲染将初始纹理与三维模型渲染至目标视角,通过与真实图片做损失,获得期望梯度以更新相机位姿。然后重新初始化纹理图,再将纹理图设置为优化变量利用对抗生成网络重合成貌似真实的纹理。经过纹理重合成,我们可以得到全局一致,高保真纹理模型。\par
提出联合优化算法同时对RGB-D重建的相机位姿、几何细节和纹理细节进行联合优化,有效解决纹理重建过程中存在的几何、位姿误差以及各种噪声。在工作一基础上我们利用渲染优化三维模型顶点位置而且然后对于三维模型中纹理丰富区域进行质心细分,以增加顶点和面片数目对几何细节增强,引导顶点位置矫正。然后我们提出交替优化策略分别对几何、纹理、位姿进行优化获取全局最优解。算法最后可以获得高保真的几何和纹理模型。
%==============================
\section{未来展望}
由于在VR/AR、动画、视频游戏等领域有着广泛的应用前景,构建具有高保真纹理和几何图形的真实世界3D对象和场景一直是一个重要的问题。现阶段本文算法在纹理优化领域取得了一定的研究成果,但是还有很多问题需要进行解决。\par
目前纹理优化领域所使用数据集均为,论文作者自己拍摄,一致缺乏第三方标准的公共数据集。拍摄场景、设备、模型重建质量均有所差别,这给我们评估纹理优化算法带来挑战。下一步可以构建一些标准数据集供他人使用。\par
纹理优化算法以传统为主,深度学习结合的很少。当前基于数据驱动的算法,存在较大缺陷。第一,只适合于小场景的纹理优化,对于大型场景的表现不如传统算法。第二,在无纹理或者颜色值单一的场景中纹理优化效果会显著下降。未来可以提出更加鲁棒性的算法,而且利用神经网络强大学习能力,对于材质,灯光分别进行建模以更加贴近真实世界。\par
纹理依赖于几何模型存在,理论上几何模型重建越精确,纹理优化结果越好。不幸的是由于重建算法本身的缺陷或者物理遮挡使得几何模型会发生残缺现象。这破坏了几何模型的拓扑结构,这为纹理优化带来新的挑战。未来可以先对几何模型缺失的地方进行补全,从而为后续获取高质量纹理模型提供保障。

% 参考文献
% 2015版国标GBT7714-2015
% 2005版国标GBT7714-2005

% 用.bib文件形式引入参考文献
%\Bib{bst/GBT7714-2015}{ref}

%手动添加参考文献
\bib
\begin{thebibliography}{00}


\bibitem{RichardNewcombe2011KinectFusionRD} Newcombe R A, Izadi S, Hilliges O, et al. Kinectfusion: Real-time dense surface mapping and tracking[C].2011 10th IEEE international symposium on mixed and augmented reality. Ieee, 2011: 127-136.

\bibitem{ThomasWhelan2012KintinuousSE}
Whelan T, Kaess M, Fallon M, et al. Kintinuous: Spatially extended kinectfusion[J]. 2012.

\bibitem{ThomasWhelan2015ElasticFusionDS}Whelan T, Leutenegger S, Salas-Moreno R, et al. ElasticFusion: Dense SLAM without a pose graph[C]. Robotics: Science and Systems, 2015.


\bibitem{VictorAdrianPrisacariu2017InfiniTAMVA} Prisacariu V A, Kähler O, Golodetz S, et al. Infinitam v3: A framework for large-scale 3d reconstruction with loop closure[J]. arXiv preprint arXiv:1708.00783, 2017.

\bibitem{AngelaDai2016BundleFusionRG}Dai A, Nießner M, Zollhöfer M, et al. Bundlefusion: Real-time globally consistent 3d reconstruction using on-the-fly surface reintegration[J]. ACM Transactions on Graphics (ToG), 2017, 36(4): 1.

\bibitem{DejanAzinovic2021NeuralRS} Azinović D, Martin-Brualla R, Goldman D B, et al. Neural rgb-d surface reconstruction[C]. Proceedings of the IEEE/CVF Conference on Computer Vision and Pattern Recognition. 2022: 6290-6301.

\bibitem{fuyanping}付燕平.\enspace 面向RGB-D相机高保真三维重建与纹理映射研究[D]. \enspace 武汉大学,\enspace 2020. DOI:10.27379/d.cnki.gwhdu.2020.000647.
 
\bibitem{tongliyang}童立靖,\enspace 杨鑫坡.\enspace 基于相机标定的纹理映射方法[J].数字技术与应用,\enspace 2022,\enspace 40(09):1-3. DOI:10.19695/j.cnki.cn12-1369.2022.09.01.

\bibitem{maolibo}毛力波.\enspace 实景三维纹理映射与增强方法研究[D].\enspace 山东建筑大学, \! 2022. \enspace DOI:10.27273/d.cnki.gsajc.2022.000004.

\bibitem{lempitsky2007seamless}Lempitsky V, Ivanov D. Seamless mosaicing of image-based texture maps[C]. 2007 IEEE conference on computer vision and pattern recognition. IEEE, 2007: 1-6.


\bibitem{boykov2001fast}Boykov Y, Veksler O, Zabih R. Fast approximate energy minimization via graph cuts[J]. IEEE Transactions on pattern analysis and machine intelligence, 2001, 23(11): 1222-1239.

\bibitem{waechter2014let}Waechter M, Moehrle N, Goesele M. Let there be color! Large-scale texturing of 3D reconstructions[C]. Computer Vision–ECCV 2014: 13th European Conference, Zurich, Switzerland, September 6-12, 2014, Proceedings, Part V 13. Springer International Publishing, 2014: 836-850.


\bibitem{PrezPatrick2003PoissonIE}Pérez P, Gangnet M, Blake A. Poisson image editing[M]. ACM SIGGRAPH 2003 Papers. 2003: 313-318.

\bibitem{fu2018texture}Fu Y, Yan Q, Yang L, et al. Texture mapping for 3d reconstruction with rgb-d sensor[C]. Proceedings of the IEEE conference on computer vision and pattern recognition. 2018: 4645-4653.

\bibitem{franken2005minimizing}Franken T, Dellepiane M, Ganovelli F, et al. Minimizing user intervention in registering 2D images to 3D models[J]. The Visual Computer, 2005, 21: 619-628.

\bibitem{eisemann2008floating}Eisemann M, De Decker B, Magnor M, et al. Floating textures[C]. Computer graphics forum. Oxford, UK: Blackwell Publishing Ltd, 2008, 27(2): 409-418.

\bibitem{dellepiane2011flow}Dellepiane M, Marroquim R, Callieri M, et al. Flow-based local optimization for image-to-geometry projection[J]. IEEE Transactions on Visualization and Computer Graphics, 2011, 18(3): 463-474.

\bibitem{wu20083d}Wu C, Clipp B, Li X, et al. 3D model matching with viewpoint-invariant patches (VIP)[C]. 2008 IEEE Conference on Computer Vision and Pattern Recognition. IEEE, 2008: 1-8.

\bibitem{aganj2010multi}Aganj E, Monasse P, Keriven R. Multi-view texturing of imprecise mesh[C]. Computer Vision–ACCV 2009: 9th Asian Conference on Computer Vision, Xi’an, September 23-27, 2009, Revised Selected Papers, Part II 9. Springer Berlin Heidelberg, 2010: 468-476.

\bibitem{zhou2014color}Zhou Q Y, Koltun V. Color map optimization for 3d reconstruction with consumer depth cameras[J]. ACM Transactions on Graphics (ToG), 2014, 33(4): 1-10.

\bibitem{Barnes:2009:PAR}Barnes C, Shechtman E, Finkelstein A, et al. PatchMatch: A randomized correspondence algorithm for structural image editing[J]. ACM Trans. Graph., 2009, 28(3): 24.

\bibitem{bi2017patch}Bi S, Kalantari N K, Ramamoorthi R. Patch-based optimization for image-based texture mapping[J]. ACM Trans. Graph., 2017, 36(4): 106:1-106:11.

\bibitem{fu2021seamless}Fu Y, Yan Q, Liao J, et al. Seamless texture optimization for RGB-D reconstruction[J]. IEEE Transactions on Visualization and Computer Graphics, 2021.

\bibitem{RobertMaier2017Intrinsic3DH3}Maier R, Kim K, Cremers D, et al. Intrinsic3D: High-quality 3D reconstruction by joint appearance and geometry optimization with spatially-varying lighting[C]. Proceedings of the IEEE international conference on computer vision. 2017: 3114-3122. 

\bibitem{YanpingFu2020JointTA}Fu Y, Yan Q, Liao J, et al. Joint texture and geometry optimization for RGB-D reconstruction[C]. Proceedings of the IEEE/CVF Conference on Computer Vision and Pattern Recognition. 2020: 5950-5959.

\bibitem{NIPS2014_5ca3e9b1}Goodfellow I, Pouget-Abadie J, Mirza M, et al. Generative adversarial networks[J]. Communications of the ACM, 2020, 63(11): 139-144.

\bibitem{isola2017image}Isola P, Zhu J Y, Zhou T, et al. Image-to-image translation with conditional adversarial networks[C]. Proceedings of the IEEE conference on computer vision and pattern recognition. 2017: 1125-1134.

\bibitem{zhu2017unpaired}Zhu J Y, Park T, Isola P, et al. Unpaired image-to-image translation using cycle-consistent adversarial networks[C]. Proceedings of the IEEE international conference on computer vision. 2017: 2223-2232.

\bibitem{JingweiHuang2020AdversarialTO}Huang J, Thies J, Dai A, et al. Adversarial texture optimization from rgb-d scans[C]. Proceedings of the IEEE/CVF Conference on Computer Vision and Pattern Recognition. 2020: 1559-1568.


\bibitem{9705143}Zhang J, Wan Z, Liao J. Adaptive joint optimization for 3D reconstruction with differentiable rendering[J]. IEEE Transactions on Visualization and Computer Graphics, 2022.


\bibitem{chanmonteiro2020pi-GAN}Chan E R, Monteiro M, Kellnhofer P, et al. pi-gan: Periodic implicit generative adversarial networks for 3d-aware image synthesis[C]. Proceedings of the IEEE/CVF conference on computer vision and pattern recognition. 2021: 5799-5809.


\bibitem{888718}Zhang Z. A flexible new technique for camera calibration[J]. IEEE Transactions on pattern analysis and machine intelligence, 2000, 22(11): 1330-1334.

\bibitem{nguyen2018rendernet}Nguyen-Phuoc T H, Li C, Balaban S, et al. Rendernet: A deep convolutional network for differentiable rendering from 3d shapes[J]. Advances in neural information processing systems, 2018, 31.

\bibitem{BrianCurless1996AVM}Curless B, Levoy M. A volumetric method for building complex models from range images[C]. Proceedings of the 23rd annual conference on Computer graphics and interactive techniques. 1996: 303-312.

\bibitem{511}Levoy M. Display of surfaces from volume data[J]. IEEE Computer graphics and Applications, 1988, 8(3): 29-37.

\bibitem{lorensen1987marching}Lorensen W E, Cline H E. Marching cubes: A high resolution 3D surface construction algorithm[J]. ACM siggraph computer graphics, 1987, 21(4): 163-169.


\bibitem{LarsMescheder2018OccupancyNL}Mescheder L, Oechsle M, Niemeyer M, et al. Occupancy networks: Learning 3d reconstruction in function space[C]. Proceedings of the IEEE/CVF conference on computer vision and pattern recognition. 2019: 4460-4470.

\bibitem{mildenhall2021nerf}Mildenhall B, Srinivasan P P, Tancik M, et al. Nerf: Representing scenes as neural radiance fields for view synthesis[J]. Communications of the ACM, 2021, 65(1): 99-106.

\bibitem{park2019deepsdf}Park J J, Florence P, Straub J, et al. Deepsdf: Learning continuous signed distance functions for shape representation[C]. Proceedings of the IEEE/CVF conference on computer vision and pattern recognition. 2019: 165-174.

\bibitem{hart1996sphere}Hart J C. Sphere tracing: A geometric method for the antialiased ray tracing of implicit surfaces[J]. The Visual Computer, 1996, 12(10): 527-545.

\bibitem{Deepvoxels}Sitzmann V, Thies J, Heide F, et al. Deepvoxels: Learning persistent 3d feature embeddings[C]. Proceedings of the IEEE/CVF Conference on Computer Vision and Pattern Recognition. 2019: 2437-2446.

\bibitem{qi2017pointnet}Qi C R, Su H, Mo K, et al. Pointnet: Deep learning on point sets for 3d classification and segmentation[C]. Proceedings of the IEEE conference on computer vision and pattern recognition. 2017: 652-660.


\bibitem{qi2017pointnet++}Qi C R, Yi L, Su H, et al. Pointnet++: Deep hierarchical feature learning on point sets in a metric space[J]. Advances in neural information processing systems, 2017, 30.

\bibitem{kazhdan2006poisson}Kazhdan M, Bolitho M, Hoppe H. Poisson surface reconstruction[C]. Proceedings of the fourth Eurographics symposium on Geometry processing. 2006, 7: 0.

\bibitem {LocalChapterEvents:ItalChap:ItalianChapConf2008:129-136}Cignoni P, Callieri M, Corsini M, et al. Meshlab: an open-source mesh processing tool[C]. Eurographics Italian chapter conference. 2008, 2008: 129-136.

\bibitem{ShichenLiu2019SoftRA}Liu S, Li T, Chen W, et al. Soft rasterizer: A differentiable renderer for image-based 3d reasoning[C]. Proceedings of the IEEE/CVF International Conference on Computer Vision. 2019: 7708-7717.

\bibitem{shapenet2015}Chang A X, Funkhouser T, Guibas L, et al. Shapenet: An information-rich 3d model repository[J]. arXiv preprint arXiv:1512.03012, 2015.

\bibitem{greene1986environment}Greene N. Environment mapping and other applications of world projections[J]. IEEE computer graphics and Applications, 1986, 6(11): 21-29.

\bibitem{tarini2017rethinking} Tarini M, Yuksel C, Lefebvre S. Rethinking texture mapping[M]. ACM SIGGRAPH 2017 Courses. 2017: 1-139.

\bibitem{yuksel2019rethinking}Yuksel C, Lefebvre S, Tarini M. Rethinking texture mapping[C]. Computer Graphics Forum. 2019, 38(2): 535-551.

\bibitem{HiroharuKato2017Neural3M}Kato H, Ushiku Y, Harada T. Neural 3d mesh renderer[C]. Proceedings of the IEEE conference on computer vision and pattern recognition. 2018: 3907-3916.

\bibitem{yariv2020multiview}Yariv L, Kasten Y, Moran D, et al. Multiview neural surface reconstruction by disentangling geometry and appearance[J]. Advances in Neural Information Processing Systems, 2020, 33: 2492-2502.

\bibitem{zhang2020nerf++}Zhang K, Riegler G, Snavely N, et al. Nerf++: Analyzing and improving neural radiance fields[J]. arXiv preprint arXiv:2010.07492, 2020.

\bibitem{tewari2020state}Tewari A, Fried O, Thies J, et al. State of the art on neural rendering[C]. Computer Graphics Forum. 2020, 39(2): 701-727.


\bibitem{kajiya1986rendering}Kajiya J T. The rendering equation[C]. Proceedings of the 13th annual conference on Computer graphics and interactive techniques. 1986: 143-150.

\bibitem{blinn1977models}Blinn J F. Models of light reflection for computer synthesized pictures[C]. Proceedings of the 4th annual conference on Computer graphics and interactive techniques. 1977: 192-198.

\bibitem{niemeyer2020differentiable}Niemeyer M, Mescheder L, Oechsle M, et al. Differentiable volumetric rendering: Learning implicit 3d representations without 3d supervision[C]. Proceedings of the IEEE/CVF Conference on Computer Vision and Pattern Recognition. 2020: 3504-3515.

\bibitem{jiang2020sdfdiff}Jiang Y, Ji D, Han Z, et al. Sdfdiff: Differentiable rendering of signed distance fields for 3d shape optimization[C]. Proceedings of the IEEE/CVF Conference on Computer Vision and Pattern Recognition. 2020: 1251-1261.

\bibitem{liu2020dist}Liu S, Zhang Y, Peng S, et al. Dist: Rendering deep implicit signed distance function with differentiable sphere tracing[C]. Proceedings of the IEEE/CVF Conference on Computer Vision and Pattern Recognition. 2020: 2019-2028.

\bibitem{wiles2020synsin}Wiles O, Gkioxari G, Szeliski R, et al. Synsin: End-to-end view synthesis from a single image[C]. Proceedings of the IEEE/CVF Conference on Computer Vision and Pattern Recognition. 2020: 7467-7477.

\bibitem{lassner2021pulsar}Lassner C, Zollhofer M. Pulsar: Efficient sphere-based neural rendering[C]. Proceedings of the IEEE/CVF Conference on Computer Vision and Pattern Recognition. 2021: 1440-1449.

\bibitem{ravi2020pytorch3d}Ravi N, Reizenstein J, Novotny D, et al. Accelerating 3d deep learning with pytorch3d[J]. arXiv preprint arXiv:2007.08501, 2020.

\bibitem{SungjoonChoi2015RobustRO}Choi S, Zhou Q Y, Koltun V. Robust reconstruction of indoor scenes[C]. Proceedings of the IEEE Conference on Computer Vision and Pattern Recognition. 2015: 5556-5565.

 \bibitem{MatthiasNiener2013Realtime3R}Nießner M, Zollhöfer M, Izadi S, et al. Real-time 3D reconstruction at scale using voxel hashing[J]. ACM Transactions on Graphics (ToG), 2013, 32(6): 1-11.


\bibitem{sola2018micro}Sola J, Deray J, Atchuthan D. A micro Lie theory for state estimation in robotics[J]. arXiv preprint arXiv:1812.01537, 2018.


\bibitem{liu2020general}Liu S, Li T, Chen W, et al. A general differentiable mesh renderer for image-based 3D reasoning[J]. IEEE Transactions on Pattern Analysis and Machine Intelligence, 2020, 44(1): 50-62.

\bibitem{huang20173dlite}Huang J, Dai A, Guibas L J, et al. 3Dlite: towards commodity 3D scanning for content creation[J]. ACM Trans. Graph., 2017, 36(6): 203:1-203:14.

\bibitem{MatthewLoper2014OpenDRAA}Loper M M, Black M J. OpenDR: An approximate differentiable renderer[C]. Computer Vision–ECCV 2014: 13th European Conference, Zurich, Switzerland, September 6-12, 2014, Proceedings, Part VII 13. Springer International Publishing, 2014: 154-169.

\bibitem{niessner2013real}Nießner M, Zollhöfer M, Izadi S, et al. Real-time 3D reconstruction at scale using voxel hashing[J]. ACM Transactions on Graphics (ToG), 2013, 32(6): 1-11.

\bibitem{KyleGenova2018UnsupervisedTF}Genova K, Cole F, Maschinot A, et al. Unsupervised training for 3d morphable model regression[C]. Proceedings of the IEEE Conference on Computer Vision and Pattern Recognition. 2018: 8377-8386.

\bibitem{HelgeRhodin2015AVS}Rhodin H, Robertini N, Richardt C, et al. A versatile scene model with differentiable visibility applied to generative pose estimation[C]. Proceedings of the IEEE International Conference on Computer Vision. 2015: 765-773.

\bibitem{chen2019_dibr}Lyu H, Sha N, Qin S, et al. Advances in neural information processing systems[J]. Advances in neural information processing systems, 2019, 32.

\bibitem{fu2021auto}Fu L, Zhou C, Guo Q, et al. Auto-exposure fusion for single-image shadow removal[C]. Proceedings of the IEEE/CVF conference on computer vision and pattern recognition. 2021: 10571-10580.

\bibitem{fu2021multi}Fu G, Zhang Q, Zhu L, et al. A multi-task network for joint specular highlight detection and removal[C]. Proceedings of the IEEE/CVF Conference on Computer Vision and Pattern Recognition. 2021: 7752-7761.

\bibitem{sauer2023stylegan}Sauer A, Karras T, Laine S, et al. Stylegan-t: Unlocking the power of gans for fast large-scale text-to-image synthesis[J]. arXiv preprint arXiv:2301.09515, 2023.

\bibitem{yu2018generative}Yu J, Lin Z, Yang J, et al. Generative image inpainting with contextual attention[C]. Proceedings of the IEEE conference on computer vision and pattern recognition. 2018: 5505-5514.

\bibitem{wang2018high}Wang T C, Liu M Y, Zhu J Y, et al. High-resolution image synthesis and semantic manipulation with conditional gans[C]. Proceedings of the IEEE conference on computer vision and pattern recognition. 2018: 8798-8807.


\bibitem{LongYang2018SurfaceRV}Yang L, Yan Q, Fu Y, et al. Surface reconstruction via fusing sparse-sequence of depth images[J]. IEEE transactions on visualization and computer graphics, 2017, 24(2): 1190-1203.



\bibitem{FrederiqueCrete2007TheBE}Crete F, Dolmiere T, Ladret P, et al. The blur effect: perception and estimation with a new no-reference perceptual blur metric[C]. Human vision and electronic imaging XII. SPIE, 2007, 6492: 196-206.


\bibitem{besl1992method}Besl P J, McKay N D. Method for registration of 3-D shapes[C]. Sensor fusion IV: control paradigms and data structures. Spie, 1992, 1611: 586-606.


\bibitem{chen1992object}Chen Y, Medioni G. Object modelling by registration of multiple range images[J]. Image and vision computing, 1992, 10(3): 145-155.

\bibitem{wedderburn1974quasi}Wedderburn R W M. Quasi-likelihood functions, generalized linear models, and the Gauss—Newton method[J]. Biometrika, 1974, 61(3): 439-447.


\bibitem{liu2018intriguing}Liu R, Lehman J, Molino P, et al. An intriguing failing of convolutional neural networks and the coordconv solution[J]. Advances in neural information processing systems, 2018, 31.

\bibitem{de2003improved}De Boer J F, Cense B, Park B H, et al. Improved signal-to-noise ratio in spectral-domain compared with time-domain optical coherence tomography[J]. Optics letters, 2003, 28(21): 2067-2069.

\bibitem{brunet2011mathematical}Brunet D, Vrscay E R, Wang Z. On the mathematical properties of the structural similarity index[J]. IEEE Transactions on Image Processing, 2011, 21(4): 1488-1499.

\bibitem{zhang2018unreasonable}Zhang R, Isola P, Efros A A, et al. The unreasonable effectiveness of deep features as a perceptual metric[C]. Proceedings of the IEEE conference on computer vision and pattern recognition. 2018: 586-595.

\bibitem{kingma2014adam}Kingma D P, Ba J. Adam: A method for stochastic optimization[J]. arXiv preprint arXiv:1412.6980, 2014.

\bibitem{newcombe2015dynamicfusion} Newcombe R A, Fox D, Seitz S M. Dynamicfusion: Reconstruction and tracking of non-rigid scenes in real-time[C]. Proceedings of the IEEE conference on computer vision and pattern recognition. 2015: 343-352.

\bibitem{innmann2016volumedeform} Innmann M, Zollhöfer M, Nießner M, et al. Volumedeform: Real-time volumetric non-rigid reconstruction[C]. Computer Vision–ECCV 2016: 14th European Conference, Amsterdam, The Netherlands, October 11-14, 2016, Proceedings, Part VIII 14. Springer International Publishing, 2016: 362-379.

\bibitem{yu2017bodyfusion}Yu T, Guo K, Xu F, et al. Bodyfusion: Real-time capture of human motion and surface geometry using a single depth camera[C]. Proceedings of the IEEE International Conference on Computer Vision. 2017: 910-919.

\bibitem{yu2018doublefusion}Yu T, Zheng Z, Guo K, et al. Doublefusion: Real-time capture of human performances with inner body shapes from a single depth sensor[C]. Proceedings of the IEEE conference on computer vision and pattern recognition. 2018: 7287-7296.

\bibitem{xu2019unstructuredfusion}Xu L, Su Z, Han L, et al. Unstructuredfusion: realtime 4d geometry and texture reconstruction using commercial rgbd cameras[J]. IEEE transactions on pattern analysis and machine intelligence, 2019, 42(10): 2508-2522.

\bibitem{su2020robustfusion} Su Z, Xu L, Zheng Z, et al. Robustfusion: Human volumetric capture with data-driven visual cues using a rgbd camera[C]. Computer Vision–ECCV 2020: 16th European Conference, Glasgow, UK, August 23–28, 2020, Proceedings, Part IV 16. Springer International Publishing, 2020: 246-264.


\bibitem{slavcheva2017killingfusion}Slavcheva M, Baust M, Cremers D, et al. Killingfusion: Non-rigid 3d reconstruction without correspondences[C]. Proceedings of the IEEE Conference on Computer Vision and Pattern Recognition. 2017: 1386-1395.

\bibitem{gao2019surfelwarp}Gao W, Tedrake R. Surfelwarp: Efficient non-volumetric single view dynamic reconstruction[J]. arXiv preprint arXiv:1904.13073, 2019.

\bibitem{dou2016fusion4d}Dou M, Khamis S, Degtyarev Y, et al. Fusion4d: Real-time performance capture of challenging scenes[J]. ACM Transactions on Graphics (ToG), 2016, 35(4): 1-13.

\bibitem{ChengleiWu2014RealtimeSR}Wu C, Zollhöfer M, Nießner M, et al. Real-time shading-based refinement for consumer depth cameras[J]. ACM Transactions on Graphics (ToG), 2014, 33(6): 1-10.

\bibitem{HiroharuKato2020DifferentiableRA}Kato H, Beker D, Morariu M, et al. Differentiable rendering: A survey[J]. arXiv preprint arXiv:2006.12057, 2020.

\bibitem{agarap2018deep}Agarap A F. Deep learning using rectified linear units (relu)[J]. arXiv preprint arXiv:1803.08375, 2018.

\bibitem{jiang2021focal}Jiang L, Dai B, Wu W, et al. Focal frequency loss for image reconstruction and synthesis[C]. Proceedings of the IEEE/CVF International Conference on Computer Vision. 2021: 13919-13929.
\bibitem{Zhou2018}Zhou Q Y, Park J, Koltun V. Open3D: A modern library for 3D data processing[J]. arXiv preprint arXiv:1801.09847, 2018.





%------------------------------------------未引用--------------------------
% \bibitem{dosovitskiy2016generating}Dosovitskiy A, Brox T. Generating images with perceptual similarity metrics based on deep networks[J]. Advances in neural information processing systems, 2016, 29.
% \bibitem{dessein2014seamless}Dessein A, Smith W A P, Wilson R C, et al. Seamless texture stitching on a 3D mesh by poisson blending in patches[C]. 2014 IEEE International Conference on Image Processing (ICIP). IEEE, 2014: 2031-2035.
% \bibitem{johnson2016perceptual}Johnson J, Alahi A, Fei-Fei L. Perceptual losses for real-time style transfer and super-resolution[C]. Computer Vision–ECCV 2016: 14th European Conference, Amsterdam, The Netherlands, October 11-14, 2016, Proceedings, Part II 14. Springer International Publishing, 2016: 694-711.
% \bibitem{karras2017progressive}Karras T, Aila T, Laine S, et al. Progressive growing of gans for improved quality, stability, and variation[J]. arXiv preprint arXiv:1710.10196, 2017.
% \bibitem{vu2011bf}Vu C T, Phan T D, Chandler D M. ${\bf S} _ {3} $: a spectral and spatial measure of local perceived sharpness in natural images[J]. IEEE transactions on image processing, 2011, 21(3): 934-945.
% \bibitem{simakov2008summarizing}Simakov D, Caspi Y, Shechtman E, et al. Summarizing visual data using bidirectional similarity[C]. 2008 IEEE Conference on Computer Vision and Pattern Recognition. IEEE, 2008: 1-8.
% \bibitem{KaolinLibrary}Jatavallabhula K M, Smith E, Lafleche J F, et al. Kaolin: A pytorch library for accelerating 3d deep learning research[J]. arXiv preprint arXiv:1911.05063, 2019.
% \bibitem{fuyanping}[1]付燕平. 面向RGB-D相机高保真三维重建与纹理映射研究[D].武汉大学,2020.DOI:10.27379/d.cnki.gwhdu.2020.000647.
% \bibitem{tongliyang}[1]童立靖,杨鑫坡.基于相机标定的纹理映射方法[J].数字技术与应用,2022,40(09):1-3.DOI:10.19695/j.cnki.cn12-1369.2022.09.01.

% \bibitem{maolibo}[1]毛力波. 实景三维纹理映射与增强方法研究[D].山东建筑大学,2022.DOI:10.27273/d.cnki.gsajc.2022.000004.


\end{thebibliography}



















  
% 附录
% % !TeX root = ../Template.tex
% [附录]
\appendix

下列内容可以作为附录:

\begin{enumerate}[label=\arabic*)]
\item 为了整篇论文材料的完整,但编入正文又有损于编排的条理和逻辑性,这一材料包括比正文更为详尽的信息、研究方法和技术更深入的叙述,建议可以阅读的参考文献题录,对了解正文内容有用的补充信息等;
\item 由于篇幅过大或取材于复制品而不便于编入正文的材料;
\item 不便于编入正文的罕见的珍贵或需要特别保密的技术细节和详细方案(这中情况可单列成册);
\item 对一般读者并非必要阅读,但对专业同行有参考价值的资料;
\item 某些重要的原始数据、过长的数学推导、计算程序、框图、结构图、注释、统计表、计算机打印输出文件等。
\end{enumerate}

\par * 嗯,自由发挥吧 * \par

% 攻读学位期间成果
% !TeX root = ../Template.tex
% [攻读学位期间取得的成果]

\achievement
\begin{enumerate}[leftmargin =0.7cm]
\renewcommand{\labelenumi}{[\theenumi]}
\item  一种基于几何先验与shading线索的三维重建方法(专利,初审合格)
\item Neural Reconstruction of Indoor Scenes using Geometric and Shading Cues(论文,在审)
\item 本人第一作者、导师通讯作者。 3D Reconstruction and Texture Optimization Based on Differentiable Rendering[C]. In 2023 8th International Conference on Intelligent Computing and Signal Processing (ICSP 2023)(已录用)

\end{enumerate}



% 攻读学位参与的科研项目
% !TeX root = ../Template.tex
% [攻读学位期间参与的科研项目]
\program

% \noindent[1] 国家自然科学基金面上项目,项目名称:基于进化多目标优化的排序学习算法研究,项目编号:62076001

% \noindent[2] 教育部人文青年基金,项目名称:大数据环境下基于排序学习的高精度检索模型研究,项目编号:18YJC870004



% (1)安徽省教育厅,安徽高校协同创新,GXXT-2022-029,基于三维视觉的MRI头动实时定量监测和MRI图像重建,2022-07 至 2024-07,100万元,在研,主持
\noindent[1] CCF-腾讯犀牛鸟科研基金,技术开发,CCF-Tencent RAGR20210117,面向移动终端基于RGB-D 相机的人体三维重建与纹理优化,2021-10 至 2022-10,15万元,本人在研,导师主持



% 致谢
%% !TeX root = ../Template.tex
% [致谢]
\acknowledgments

在这里,我想对我在硕士阶段所有支持和帮助过我的人表示衷心的感谢。\par

首先,我要感谢我的导师,付燕平教授和赵海峰老师,他们不仅给了我很多关于科学研究的指导和建议,还帮助我克服了很多困难。我特别感谢他们在我的研究中给予的大力支持,使我的研究能够顺利完成。

此外,我还要感谢我的同学们,他们在我研究过程中给予了很多宝贵的帮助。他们与我分享了自己的知识和经验,使我更好地完成了研究工作。

最后,我还要感谢我的家人,他们在我研究过程中给予了很多关爱和支持。他们不仅帮助我解决了很多家庭问题,还鼓励我继续努力。

因此,我再次对所有帮助过我的人表示衷心的感谢。如果没有他们的帮助,我的研究工作不可能顺利完成。
\par 

% 作者简介
%% !TeX root = ../Template.tex
% [作者简介]
\biography
博士学位论文应该提供作者简介,主要包括:姓名、性别、出生年月日、民族、出生的;简要学历、工作经历(职务);以及攻读博士学位期间获得的其他奖项(除攻读学位期间取得的研究成果之外)。

\par * 嗯,“硕士学位论文无此项”,《手册》上是这么说的 * \par

\end{document} 